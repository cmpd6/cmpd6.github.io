% Options for packages loaded elsewhere
\PassOptionsToPackage{unicode}{hyperref}
\PassOptionsToPackage{hyphens}{url}
\PassOptionsToPackage{dvipsnames,svgnames,x11names}{xcolor}
%
\documentclass[
  letterpaper,
  DIV=11,
  numbers=noendperiod]{scrartcl}

\usepackage{amsmath,amssymb}
\usepackage{lmodern}
\usepackage{iftex}
\ifPDFTeX
  \usepackage[T1]{fontenc}
  \usepackage[utf8]{inputenc}
  \usepackage{textcomp} % provide euro and other symbols
\else % if luatex or xetex
  \usepackage{unicode-math}
  \defaultfontfeatures{Scale=MatchLowercase}
  \defaultfontfeatures[\rmfamily]{Ligatures=TeX,Scale=1}
\fi
% Use upquote if available, for straight quotes in verbatim environments
\IfFileExists{upquote.sty}{\usepackage{upquote}}{}
\IfFileExists{microtype.sty}{% use microtype if available
  \usepackage[]{microtype}
  \UseMicrotypeSet[protrusion]{basicmath} % disable protrusion for tt fonts
}{}
\makeatletter
\@ifundefined{KOMAClassName}{% if non-KOMA class
  \IfFileExists{parskip.sty}{%
    \usepackage{parskip}
  }{% else
    \setlength{\parindent}{0pt}
    \setlength{\parskip}{6pt plus 2pt minus 1pt}}
}{% if KOMA class
  \KOMAoptions{parskip=half}}
\makeatother
\usepackage{xcolor}
\usepackage[top=15mm,bottom=20mm,left=15mm,right=15mm]{geometry}
\setlength{\emergencystretch}{3em} % prevent overfull lines
\setcounter{secnumdepth}{-\maxdimen} % remove section numbering
% Make \paragraph and \subparagraph free-standing
\ifx\paragraph\undefined\else
  \let\oldparagraph\paragraph
  \renewcommand{\paragraph}[1]{\oldparagraph{#1}\mbox{}}
\fi
\ifx\subparagraph\undefined\else
  \let\oldsubparagraph\subparagraph
  \renewcommand{\subparagraph}[1]{\oldsubparagraph{#1}\mbox{}}
\fi


\providecommand{\tightlist}{%
  \setlength{\itemsep}{0pt}\setlength{\parskip}{0pt}}\usepackage{longtable,booktabs,array}
\usepackage{calc} % for calculating minipage widths
% Correct order of tables after \paragraph or \subparagraph
\usepackage{etoolbox}
\makeatletter
\patchcmd\longtable{\par}{\if@noskipsec\mbox{}\fi\par}{}{}
\makeatother
% Allow footnotes in longtable head/foot
\IfFileExists{footnotehyper.sty}{\usepackage{footnotehyper}}{\usepackage{footnote}}
\makesavenoteenv{longtable}
\usepackage{graphicx}
\makeatletter
\def\maxwidth{\ifdim\Gin@nat@width>\linewidth\linewidth\else\Gin@nat@width\fi}
\def\maxheight{\ifdim\Gin@nat@height>\textheight\textheight\else\Gin@nat@height\fi}
\makeatother
% Scale images if necessary, so that they will not overflow the page
% margins by default, and it is still possible to overwrite the defaults
% using explicit options in \includegraphics[width, height, ...]{}
\setkeys{Gin}{width=\maxwidth,height=\maxheight,keepaspectratio}
% Set default figure placement to htbp
\makeatletter
\def\fps@figure{htbp}
\makeatother

\newenvironment{absolutelynopagebreak}
  {\par\nobreak\vfil\penalty0\vfilneg
   \vtop\bgroup}
  {\par\xdef\tpd{\the\prevdepth}\egroup
   \prevdepth=\tpd}
\KOMAoption{captions}{tableheading}
\makeatletter
\makeatother
\makeatletter
\makeatother
\makeatletter
\@ifpackageloaded{caption}{}{\usepackage{caption}}
\AtBeginDocument{%
\ifdefined\contentsname
  \renewcommand*\contentsname{Table of contents}
\else
  \newcommand\contentsname{Table of contents}
\fi
\ifdefined\listfigurename
  \renewcommand*\listfigurename{List of Figures}
\else
  \newcommand\listfigurename{List of Figures}
\fi
\ifdefined\listtablename
  \renewcommand*\listtablename{List of Tables}
\else
  \newcommand\listtablename{List of Tables}
\fi
\ifdefined\figurename
  \renewcommand*\figurename{Figure}
\else
  \newcommand\figurename{Figure}
\fi
\ifdefined\tablename
  \renewcommand*\tablename{Table}
\else
  \newcommand\tablename{Table}
\fi
}
\@ifpackageloaded{float}{}{\usepackage{float}}
\floatstyle{ruled}
\@ifundefined{c@chapter}{\newfloat{codelisting}{h}{lop}}{\newfloat{codelisting}{h}{lop}[chapter]}
\floatname{codelisting}{Listing}
\newcommand*\listoflistings{\listof{codelisting}{List of Listings}}
\makeatother
\makeatletter
\@ifpackageloaded{caption}{}{\usepackage{caption}}
\@ifpackageloaded{subcaption}{}{\usepackage{subcaption}}
\makeatother
\makeatletter
\@ifpackageloaded{tcolorbox}{}{\usepackage[many]{tcolorbox}}
\makeatother
\makeatletter
\@ifundefined{shadecolor}{\definecolor{shadecolor}{rgb}{.97, .97, .97}}
\makeatother
\makeatletter
\makeatother
\ifLuaTeX
  \usepackage{selnolig}  % disable illegal ligatures
\fi
\IfFileExists{bookmark.sty}{\usepackage{bookmark}}{\usepackage{hyperref}}
\IfFileExists{xurl.sty}{\usepackage{xurl}}{} % add URL line breaks if available
\urlstyle{same} % disable monospaced font for URLs
\hypersetup{
  pdftitle={CMPD6 abstracts},
  colorlinks=true,
  linkcolor={blue},
  filecolor={Maroon},
  citecolor={Blue},
  urlcolor={Blue},
  pdfcreator={LaTeX via pandoc}}

\title{CMPD6 abstracts}
\author{}
\date{}

\begin{document}
\maketitle
\ifdefined\Shaded\renewenvironment{Shaded}{\begin{tcolorbox}[breakable, enhanced, frame hidden, borderline west={3pt}{0pt}{shadecolor}, boxrule=0pt, interior hidden, sharp corners]}{\end{tcolorbox}}\fi

\renewcommand*\contentsname{Table of contents}
{
\hypersetup{linkcolor=}
\setcounter{tocdepth}{3}
\tableofcontents
}
\newpage

\begin{absolutelynopagebreak}
\subsection{Azmy Ackleh - A Multiple-Strain Susceptible-Infected Model with Diffusion Formulated on the Space of Radon Measures} 

\begin{tabular}{l}
\toprule
University of Louisiana at Lafayette\\
USA\\
\bottomrule
\end{tabular}

\begin{tabular}{l}
\toprule
Minisymposium presentation\\
(Ecological and Epidemiological Models with Dispersal)\\
\bottomrule
\end{tabular}
\vskip0.5cm

  Mathematical modelling is an important tool for understanding and controlling the spread of infectious diseases. Heterogeneity, which refers to differences in factors such as demographics, behaviour, susceptibility, infectiousness, and disease severity within a population, plays a critical role in disease transmission and control. Incorporating heterogeneity into models can help researchers better understand disease spread across subpopulations and design more targeted control strategies. However, heterogeneous models can be high-dimensional and complex, leading to theoretical challenges in modelling analysis. Moreover, data collections in the field are often in the forms of aggregation, making modeling implementation challenging. In this talk, we will discuss recent developments and remaining challenges in modeling infectious diseases with a focus on heterogeneity and aggregation. The goal is to provide attendees with valuable insights into the significance of incorporating heterogeneity into models and effective ways to address associated challenges. 


\vskip\medskipamount

  \leaders\vrule width \textwidth\vskip0.4pt

  \vskip\medskipamount

  \nointerlineskip

  \pagebreak[2]

\end{absolutelynopagebreak}

\begin{absolutelynopagebreak}
\subsection{Folashade Agusto - Exploring the effects of prescribed fire and rising temperature on tick-borne diseases} 

\begin{tabular}{l}
\toprule
University of Kansas\\
USA\\
\bottomrule
\end{tabular}

\begin{tabular}{l}
\toprule
Minisymposium presentation\\
(Vector-Borne Disease Dynamics)\\
\bottomrule
\end{tabular}
\vskip0.5cm

  Mathematical modelling is an important tool for understanding and controlling the spread of infectious diseases. Heterogeneity, which refers to differences in factors such as demographics, behaviour, susceptibility, infectiousness, and disease severity within a population, plays a critical role in disease transmission and control. Incorporating heterogeneity into models can help researchers better understand disease spread across subpopulations and design more targeted control strategies. However, heterogeneous models can be high-dimensional and complex, leading to theoretical challenges in modelling analysis. Moreover, data collections in the field are often in the forms of aggregation, making modeling implementation challenging. In this talk, we will discuss recent developments and remaining challenges in modeling infectious diseases with a focus on heterogeneity and aggregation. The goal is to provide attendees with valuable insights into the significance of incorporating heterogeneity into models and effective ways to address associated challenges. 


\vskip\medskipamount

  \leaders\vrule width \textwidth\vskip0.4pt

  \vskip\medskipamount

  \nointerlineskip

  \pagebreak[2]

\end{absolutelynopagebreak}

\begin{absolutelynopagebreak}
\subsection{Folashade Agusto - From cultural practices to risky behaviors to public sentiment: Modeling human behavior and disease transmission} 

\begin{tabular}{l}
\toprule
University of Kansas\\
USA\\
\bottomrule
\end{tabular}

\begin{tabular}{l}
\toprule
Plenary presentation\\
\bottomrule
\end{tabular}
\vskip0.5cm

  Mathematical modelling is an important tool for understanding and controlling the spread of infectious diseases. Heterogeneity, which refers to differences in factors such as demographics, behaviour, susceptibility, infectiousness, and disease severity within a population, plays a critical role in disease transmission and control. Incorporating heterogeneity into models can help researchers better understand disease spread across subpopulations and design more targeted control strategies. However, heterogeneous models can be high-dimensional and complex, leading to theoretical challenges in modelling analysis. Moreover, data collections in the field are often in the forms of aggregation, making modeling implementation challenging. In this talk, we will discuss recent developments and remaining challenges in modeling infectious diseases with a focus on heterogeneity and aggregation. The goal is to provide attendees with valuable insights into the significance of incorporating heterogeneity into models and effective ways to address associated challenges. 


\vskip\medskipamount

  \leaders\vrule width \textwidth\vskip0.4pt

  \vskip\medskipamount

  \nointerlineskip

  \pagebreak[2]

\end{absolutelynopagebreak}

\begin{absolutelynopagebreak}
\subsection{Ephraim Agyingi - Modeling immune system priming: the miracle that saved Sub-Sahara Africa from COVID-19} 

\begin{tabular}{l}
\toprule
Rochester Institute of Technology, Rochester, New York\\
USA\\
\bottomrule
\end{tabular}

\begin{tabular}{l}
\toprule
Minisymposium presentation\\
(Within-host and between-host mathematical models of biological dynamics)\\
\bottomrule
\end{tabular}
\vskip0.5cm

  Mathematical modelling is an important tool for understanding and controlling the spread of infectious diseases. Heterogeneity, which refers to differences in factors such as demographics, behaviour, susceptibility, infectiousness, and disease severity within a population, plays a critical role in disease transmission and control. Incorporating heterogeneity into models can help researchers better understand disease spread across subpopulations and design more targeted control strategies. However, heterogeneous models can be high-dimensional and complex, leading to theoretical challenges in modelling analysis. Moreover, data collections in the field are often in the forms of aggregation, making modeling implementation challenging. In this talk, we will discuss recent developments and remaining challenges in modeling infectious diseases with a focus on heterogeneity and aggregation. The goal is to provide attendees with valuable insights into the significance of incorporating heterogeneity into models and effective ways to address associated challenges. 


\vskip\medskipamount

  \leaders\vrule width \textwidth\vskip0.4pt

  \vskip\medskipamount

  \nointerlineskip

  \pagebreak[2]

\end{absolutelynopagebreak}

\begin{absolutelynopagebreak}
\subsection{Vitalii Akimenko - Numerical Method for the Age-structured SIPCV Epidemic Model of Healthy cells, Dysplasia, Cervical Cancer Cells and HPV Dynamics} 

\begin{tabular}{l}
\toprule
University of Manitoba\\
Canada\\
\bottomrule
\end{tabular}

\begin{tabular}{l}
\toprule
Contributed presentation\\
\bottomrule
\end{tabular}
\vskip0.5cm

  Mathematical modelling is an important tool for understanding and controlling the spread of infectious diseases. Heterogeneity, which refers to differences in factors such as demographics, behaviour, susceptibility, infectiousness, and disease severity within a population, plays a critical role in disease transmission and control. Incorporating heterogeneity into models can help researchers better understand disease spread across subpopulations and design more targeted control strategies. However, heterogeneous models can be high-dimensional and complex, leading to theoretical challenges in modelling analysis. Moreover, data collections in the field are often in the forms of aggregation, making modeling implementation challenging. In this talk, we will discuss recent developments and remaining challenges in modeling infectious diseases with a focus on heterogeneity and aggregation. The goal is to provide attendees with valuable insights into the significance of incorporating heterogeneity into models and effective ways to address associated challenges. 


\vskip\medskipamount

  \leaders\vrule width \textwidth\vskip0.4pt

  \vskip\medskipamount

  \nointerlineskip

  \pagebreak[2]

\end{absolutelynopagebreak}

\begin{absolutelynopagebreak}
\subsection{Asami Anzai - Estimating importation cases using mobility data} 

\begin{tabular}{l}
\toprule
Kyoto University\\
Japan\\
\bottomrule
\end{tabular}

\begin{tabular}{l}
\toprule
Minisymposium presentation\\
(Real time epidemiology in various geographic scales)\\
\bottomrule
\end{tabular}
\vskip0.5cm

  Mathematical modelling is an important tool for understanding and controlling the spread of infectious diseases. Heterogeneity, which refers to differences in factors such as demographics, behaviour, susceptibility, infectiousness, and disease severity within a population, plays a critical role in disease transmission and control. Incorporating heterogeneity into models can help researchers better understand disease spread across subpopulations and design more targeted control strategies. However, heterogeneous models can be high-dimensional and complex, leading to theoretical challenges in modelling analysis. Moreover, data collections in the field are often in the forms of aggregation, making modeling implementation challenging. In this talk, we will discuss recent developments and remaining challenges in modeling infectious diseases with a focus on heterogeneity and aggregation. The goal is to provide attendees with valuable insights into the significance of incorporating heterogeneity into models and effective ways to address associated challenges. 


\vskip\medskipamount

  \leaders\vrule width \textwidth\vskip0.4pt

  \vskip\medskipamount

  \nointerlineskip

  \pagebreak[2]

\end{absolutelynopagebreak}

\begin{absolutelynopagebreak}
\subsection{Julien Arino - Role of case introductions in the community spread of infectious diseases} 

\begin{tabular}{l}
\toprule
University of Manitoba\\
Canada\\
\bottomrule
\end{tabular}

\begin{tabular}{l}
\toprule
Minisymposium presentation\\
(Recent Advances in Modelling Infectious Diseases)\\
\bottomrule
\end{tabular}
\vskip0.5cm

  Mathematical modelling is an important tool for understanding and controlling the spread of infectious diseases. Heterogeneity, which refers to differences in factors such as demographics, behaviour, susceptibility, infectiousness, and disease severity within a population, plays a critical role in disease transmission and control. Incorporating heterogeneity into models can help researchers better understand disease spread across subpopulations and design more targeted control strategies. However, heterogeneous models can be high-dimensional and complex, leading to theoretical challenges in modelling analysis. Moreover, data collections in the field are often in the forms of aggregation, making modeling implementation challenging. In this talk, we will discuss recent developments and remaining challenges in modeling infectious diseases with a focus on heterogeneity and aggregation. The goal is to provide attendees with valuable insights into the significance of incorporating heterogeneity into models and effective ways to address associated challenges. 


\vskip\medskipamount

  \leaders\vrule width \textwidth\vskip0.4pt

  \vskip\medskipamount

  \nointerlineskip

  \pagebreak[2]

\end{absolutelynopagebreak}

\begin{absolutelynopagebreak}
\subsection{Joseph Baafi - Modelling the Impact of Seasonality on Mosquito Population Dynamics: Insights for Vector Control Strategies.} 

\begin{tabular}{l}
\toprule
Memorial University of Newfoundland\\
Canada\\
\bottomrule
\end{tabular}

\begin{tabular}{l}
\toprule
Contributed presentation\\
\bottomrule
\end{tabular}
\vskip0.5cm

  Mathematical modelling is an important tool for understanding and controlling the spread of infectious diseases. Heterogeneity, which refers to differences in factors such as demographics, behaviour, susceptibility, infectiousness, and disease severity within a population, plays a critical role in disease transmission and control. Incorporating heterogeneity into models can help researchers better understand disease spread across subpopulations and design more targeted control strategies. However, heterogeneous models can be high-dimensional and complex, leading to theoretical challenges in modelling analysis. Moreover, data collections in the field are often in the forms of aggregation, making modeling implementation challenging. In this talk, we will discuss recent developments and remaining challenges in modeling infectious diseases with a focus on heterogeneity and aggregation. The goal is to provide attendees with valuable insights into the significance of incorporating heterogeneity into models and effective ways to address associated challenges. 


\vskip\medskipamount

  \leaders\vrule width \textwidth\vskip0.4pt

  \vskip\medskipamount

  \nointerlineskip

  \pagebreak[2]

\end{absolutelynopagebreak}

\begin{absolutelynopagebreak}
\subsection{Rebecca Bekker - Black Holes in TIME: the Effect of GRID Radiation on the Tumor-Immune Micro-environment} 

\begin{tabular}{l}
\toprule
H. Lee Moffitt Cancer Center and Research Institute\\
USA\\
\bottomrule
\end{tabular}

\begin{tabular}{l}
\toprule
Minisymposium presentation\\
(Within-host and between-host mathematical models of biological dynamics)\\
\bottomrule
\end{tabular}
\vskip0.5cm

  Mathematical modelling is an important tool for understanding and controlling the spread of infectious diseases. Heterogeneity, which refers to differences in factors such as demographics, behaviour, susceptibility, infectiousness, and disease severity within a population, plays a critical role in disease transmission and control. Incorporating heterogeneity into models can help researchers better understand disease spread across subpopulations and design more targeted control strategies. However, heterogeneous models can be high-dimensional and complex, leading to theoretical challenges in modelling analysis. Moreover, data collections in the field are often in the forms of aggregation, making modeling implementation challenging. In this talk, we will discuss recent developments and remaining challenges in modeling infectious diseases with a focus on heterogeneity and aggregation. The goal is to provide attendees with valuable insights into the significance of incorporating heterogeneity into models and effective ways to address associated challenges. 


\vskip\medskipamount

  \leaders\vrule width \textwidth\vskip0.4pt

  \vskip\medskipamount

  \nointerlineskip

  \pagebreak[2]

\end{absolutelynopagebreak}

\begin{absolutelynopagebreak}
\subsection{Ranjini Bhattacharya - Angiogenesis in Cancer: A Tragedy of Commons} 

\begin{tabular}{l}
\toprule
Moffitt Cancer Center\\
USA\\
\bottomrule
\end{tabular}

\begin{tabular}{l}
\toprule
Contributed presentation\\
\bottomrule
\end{tabular}
\vskip0.5cm

  Mathematical modelling is an important tool for understanding and controlling the spread of infectious diseases. Heterogeneity, which refers to differences in factors such as demographics, behaviour, susceptibility, infectiousness, and disease severity within a population, plays a critical role in disease transmission and control. Incorporating heterogeneity into models can help researchers better understand disease spread across subpopulations and design more targeted control strategies. However, heterogeneous models can be high-dimensional and complex, leading to theoretical challenges in modelling analysis. Moreover, data collections in the field are often in the forms of aggregation, making modeling implementation challenging. In this talk, we will discuss recent developments and remaining challenges in modeling infectious diseases with a focus on heterogeneity and aggregation. The goal is to provide attendees with valuable insights into the significance of incorporating heterogeneity into models and effective ways to address associated challenges. 


\vskip\medskipamount

  \leaders\vrule width \textwidth\vskip0.4pt

  \vskip\medskipamount

  \nointerlineskip

  \pagebreak[2]

\end{absolutelynopagebreak}

\begin{absolutelynopagebreak}
\subsection{Amanda Bleichrodt - Multi-model forecasts in the context of the Mpox outbreak in multiple countries (July 28th, 2022 through January 26th, 2023)} 

\begin{tabular}{l}
\toprule
Georgia State University\\
USA\\
\bottomrule
\end{tabular}

\begin{tabular}{l}
\toprule
Minisymposium presentation\\
(Real time epidemiology in various geographic scales)\\
\bottomrule
\end{tabular}
\vskip0.5cm

  Mathematical modelling is an important tool for understanding and controlling the spread of infectious diseases. Heterogeneity, which refers to differences in factors such as demographics, behaviour, susceptibility, infectiousness, and disease severity within a population, plays a critical role in disease transmission and control. Incorporating heterogeneity into models can help researchers better understand disease spread across subpopulations and design more targeted control strategies. However, heterogeneous models can be high-dimensional and complex, leading to theoretical challenges in modelling analysis. Moreover, data collections in the field are often in the forms of aggregation, making modeling implementation challenging. In this talk, we will discuss recent developments and remaining challenges in modeling infectious diseases with a focus on heterogeneity and aggregation. The goal is to provide attendees with valuable insights into the significance of incorporating heterogeneity into models and effective ways to address associated challenges. 


\vskip\medskipamount

  \leaders\vrule width \textwidth\vskip0.4pt

  \vskip\medskipamount

  \nointerlineskip

  \pagebreak[2]

\end{absolutelynopagebreak}

\begin{absolutelynopagebreak}
\subsection{Anuraag Bukkuri - Models of Resistance in State-Structured Cancer Populations} 

\begin{tabular}{l}
\toprule
Moffitt Cancer Center and Lund University\\
USA\\
\bottomrule
\end{tabular}

\begin{tabular}{l}
\toprule
Contributed presentation\\
\bottomrule
\end{tabular}
\vskip0.5cm

  Mathematical modelling is an important tool for understanding and controlling the spread of infectious diseases. Heterogeneity, which refers to differences in factors such as demographics, behaviour, susceptibility, infectiousness, and disease severity within a population, plays a critical role in disease transmission and control. Incorporating heterogeneity into models can help researchers better understand disease spread across subpopulations and design more targeted control strategies. However, heterogeneous models can be high-dimensional and complex, leading to theoretical challenges in modelling analysis. Moreover, data collections in the field are often in the forms of aggregation, making modeling implementation challenging. In this talk, we will discuss recent developments and remaining challenges in modeling infectious diseases with a focus on heterogeneity and aggregation. The goal is to provide attendees with valuable insights into the significance of incorporating heterogeneity into models and effective ways to address associated challenges. 


\vskip\medskipamount

  \leaders\vrule width \textwidth\vskip0.4pt

  \vskip\medskipamount

  \nointerlineskip

  \pagebreak[2]

\end{absolutelynopagebreak}

\begin{absolutelynopagebreak}
\subsection{Jacques Bélair - Population models with state-dependant delays} 

\begin{tabular}{l}
\toprule
Université de Montréal\\
Canada\\
\bottomrule
\end{tabular}

\begin{tabular}{l}
\toprule
Minisymposium presentation\\
(Delay-differential equations in applications)\\
\bottomrule
\end{tabular}
\vskip0.5cm

  Mathematical modelling is an important tool for understanding and controlling the spread of infectious diseases. Heterogeneity, which refers to differences in factors such as demographics, behaviour, susceptibility, infectiousness, and disease severity within a population, plays a critical role in disease transmission and control. Incorporating heterogeneity into models can help researchers better understand disease spread across subpopulations and design more targeted control strategies. However, heterogeneous models can be high-dimensional and complex, leading to theoretical challenges in modelling analysis. Moreover, data collections in the field are often in the forms of aggregation, making modeling implementation challenging. In this talk, we will discuss recent developments and remaining challenges in modeling infectious diseases with a focus on heterogeneity and aggregation. The goal is to provide attendees with valuable insights into the significance of incorporating heterogeneity into models and effective ways to address associated challenges. 


\vskip\medskipamount

  \leaders\vrule width \textwidth\vskip0.4pt

  \vskip\medskipamount

  \nointerlineskip

  \pagebreak[2]

\end{absolutelynopagebreak}

\begin{absolutelynopagebreak}
\subsection{Jacques Bélair - Modeling the use of Fangsang Shelter Hospitals in Wuhan} 

\begin{tabular}{l}
\toprule
Université de Montréal\\
Canada\\
\bottomrule
\end{tabular}

\begin{tabular}{l}
\toprule
Minisymposium presentation\\
(Recent Advances in Modelling Infectious Diseases)\\
\bottomrule
\end{tabular}
\vskip0.5cm

  Mathematical modelling is an important tool for understanding and controlling the spread of infectious diseases. Heterogeneity, which refers to differences in factors such as demographics, behaviour, susceptibility, infectiousness, and disease severity within a population, plays a critical role in disease transmission and control. Incorporating heterogeneity into models can help researchers better understand disease spread across subpopulations and design more targeted control strategies. However, heterogeneous models can be high-dimensional and complex, leading to theoretical challenges in modelling analysis. Moreover, data collections in the field are often in the forms of aggregation, making modeling implementation challenging. In this talk, we will discuss recent developments and remaining challenges in modeling infectious diseases with a focus on heterogeneity and aggregation. The goal is to provide attendees with valuable insights into the significance of incorporating heterogeneity into models and effective ways to address associated challenges. 


\vskip\medskipamount

  \leaders\vrule width \textwidth\vskip0.4pt

  \vskip\medskipamount

  \nointerlineskip

  \pagebreak[2]

\end{absolutelynopagebreak}

\begin{absolutelynopagebreak}
\subsection{Robert Stephen Cantrell - Resource Matching in Spatial Ecology and Evolutionary Advantage} 

\begin{tabular}{l}
\toprule
University of Miami\\
USA\\
\bottomrule
\end{tabular}

\begin{tabular}{l}
\toprule
Minisymposium presentation\\
(Ecological and Epidemiological Models with Dispersal)\\
\bottomrule
\end{tabular}
\vskip0.5cm

  Mathematical modelling is an important tool for understanding and controlling the spread of infectious diseases. Heterogeneity, which refers to differences in factors such as demographics, behaviour, susceptibility, infectiousness, and disease severity within a population, plays a critical role in disease transmission and control. Incorporating heterogeneity into models can help researchers better understand disease spread across subpopulations and design more targeted control strategies. However, heterogeneous models can be high-dimensional and complex, leading to theoretical challenges in modelling analysis. Moreover, data collections in the field are often in the forms of aggregation, making modeling implementation challenging. In this talk, we will discuss recent developments and remaining challenges in modeling infectious diseases with a focus on heterogeneity and aggregation. The goal is to provide attendees with valuable insights into the significance of incorporating heterogeneity into models and effective ways to address associated challenges. 


\vskip\medskipamount

  \leaders\vrule width \textwidth\vskip0.4pt

  \vskip\medskipamount

  \nointerlineskip

  \pagebreak[2]

\end{absolutelynopagebreak}

\begin{absolutelynopagebreak}
\subsection{Fabian Cardozo-Ojeda - Mathematical modeling of gene and cell therapy for HIV cure} 

\begin{tabular}{l}
\toprule
Fred Hutchinson Cancer Center\\
USA\\
\bottomrule
\end{tabular}

\begin{tabular}{l}
\toprule
Minisymposium presentation\\
(Multiscale models of infectious diseases)\\
\bottomrule
\end{tabular}
\vskip0.5cm

  Mathematical modelling is an important tool for understanding and controlling the spread of infectious diseases. Heterogeneity, which refers to differences in factors such as demographics, behaviour, susceptibility, infectiousness, and disease severity within a population, plays a critical role in disease transmission and control. Incorporating heterogeneity into models can help researchers better understand disease spread across subpopulations and design more targeted control strategies. However, heterogeneous models can be high-dimensional and complex, leading to theoretical challenges in modelling analysis. Moreover, data collections in the field are often in the forms of aggregation, making modeling implementation challenging. In this talk, we will discuss recent developments and remaining challenges in modeling infectious diseases with a focus on heterogeneity and aggregation. The goal is to provide attendees with valuable insights into the significance of incorporating heterogeneity into models and effective ways to address associated challenges. 


\vskip\medskipamount

  \leaders\vrule width \textwidth\vskip0.4pt

  \vskip\medskipamount

  \nointerlineskip

  \pagebreak[2]

\end{absolutelynopagebreak}

\begin{absolutelynopagebreak}
\subsection{Bernard Cazelles - Modeling infectious disease dynamics: the challenge of non-stationarity} 

\begin{tabular}{l}
\toprule
Sorbonne Université\\
France\\
\bottomrule
\end{tabular}

\begin{tabular}{l}
\toprule
Contributed presentation\\
\bottomrule
\end{tabular}
\vskip0.5cm

  Mathematical modelling is an important tool for understanding and controlling the spread of infectious diseases. Heterogeneity, which refers to differences in factors such as demographics, behaviour, susceptibility, infectiousness, and disease severity within a population, plays a critical role in disease transmission and control. Incorporating heterogeneity into models can help researchers better understand disease spread across subpopulations and design more targeted control strategies. However, heterogeneous models can be high-dimensional and complex, leading to theoretical challenges in modelling analysis. Moreover, data collections in the field are often in the forms of aggregation, making modeling implementation challenging. In this talk, we will discuss recent developments and remaining challenges in modeling infectious diseases with a focus on heterogeneity and aggregation. The goal is to provide attendees with valuable insights into the significance of incorporating heterogeneity into models and effective ways to address associated challenges. 


\vskip\medskipamount

  \leaders\vrule width \textwidth\vskip0.4pt

  \vskip\medskipamount

  \nointerlineskip

  \pagebreak[2]

\end{absolutelynopagebreak}

\begin{absolutelynopagebreak}
\subsection{Stanca Ciupe - Multiscale models of SARS-CoV-2 infection} 

\begin{tabular}{l}
\toprule
Virginia Tech\\
USA\\
\bottomrule
\end{tabular}

\begin{tabular}{l}
\toprule
Minisymposium presentation\\
(Multiscale models of infectious diseases)\\
\bottomrule
\end{tabular}
\vskip0.5cm

  Mathematical modelling is an important tool for understanding and controlling the spread of infectious diseases. Heterogeneity, which refers to differences in factors such as demographics, behaviour, susceptibility, infectiousness, and disease severity within a population, plays a critical role in disease transmission and control. Incorporating heterogeneity into models can help researchers better understand disease spread across subpopulations and design more targeted control strategies. However, heterogeneous models can be high-dimensional and complex, leading to theoretical challenges in modelling analysis. Moreover, data collections in the field are often in the forms of aggregation, making modeling implementation challenging. In this talk, we will discuss recent developments and remaining challenges in modeling infectious diseases with a focus on heterogeneity and aggregation. The goal is to provide attendees with valuable insights into the significance of incorporating heterogeneity into models and effective ways to address associated challenges. 


\vskip\medskipamount

  \leaders\vrule width \textwidth\vskip0.4pt

  \vskip\medskipamount

  \nointerlineskip

  \pagebreak[2]

\end{absolutelynopagebreak}

\begin{absolutelynopagebreak}
\subsection{Adriana-Stefania Ciupeanu - Dynamics of COVID-19 Variants of Concern} 

\begin{tabular}{l}
\toprule
University of Manitoba\\
Canada\\
\bottomrule
\end{tabular}

\begin{tabular}{l}
\toprule
Minisymposium presentation\\
(Recent Advances in Modelling Infectious Diseases)\\
\bottomrule
\end{tabular}
\vskip0.5cm

  Mathematical modelling is an important tool for understanding and controlling the spread of infectious diseases. Heterogeneity, which refers to differences in factors such as demographics, behaviour, susceptibility, infectiousness, and disease severity within a population, plays a critical role in disease transmission and control. Incorporating heterogeneity into models can help researchers better understand disease spread across subpopulations and design more targeted control strategies. However, heterogeneous models can be high-dimensional and complex, leading to theoretical challenges in modelling analysis. Moreover, data collections in the field are often in the forms of aggregation, making modeling implementation challenging. In this talk, we will discuss recent developments and remaining challenges in modeling infectious diseases with a focus on heterogeneity and aggregation. The goal is to provide attendees with valuable insights into the significance of incorporating heterogeneity into models and effective ways to address associated challenges. 


\vskip\medskipamount

  \leaders\vrule width \textwidth\vskip0.4pt

  \vskip\medskipamount

  \nointerlineskip

  \pagebreak[2]

\end{absolutelynopagebreak}

\begin{absolutelynopagebreak}
\subsection{Jessica Conway - Heterogeneity in HIV viral rebound} 

\begin{tabular}{l}
\toprule
Penn State\\
USA\\
\bottomrule
\end{tabular}

\begin{tabular}{l}
\toprule
Minisymposium presentation\\
(Mathematical and computational approaches to modelling immunology)\\
\bottomrule
\end{tabular}
\vskip0.5cm

  Mathematical modelling is an important tool for understanding and controlling the spread of infectious diseases. Heterogeneity, which refers to differences in factors such as demographics, behaviour, susceptibility, infectiousness, and disease severity within a population, plays a critical role in disease transmission and control. Incorporating heterogeneity into models can help researchers better understand disease spread across subpopulations and design more targeted control strategies. However, heterogeneous models can be high-dimensional and complex, leading to theoretical challenges in modelling analysis. Moreover, data collections in the field are often in the forms of aggregation, making modeling implementation challenging. In this talk, we will discuss recent developments and remaining challenges in modeling infectious diseases with a focus on heterogeneity and aggregation. The goal is to provide attendees with valuable insights into the significance of incorporating heterogeneity into models and effective ways to address associated challenges. 


\vskip\medskipamount

  \leaders\vrule width \textwidth\vskip0.4pt

  \vskip\medskipamount

  \nointerlineskip

  \pagebreak[2]

\end{absolutelynopagebreak}

\begin{absolutelynopagebreak}
\subsection{Jessica Conway - Modeling PrEP-on-demand strategies to prevent HIV transmission} 

\begin{tabular}{l}
\toprule
Penn State\\
USA\\
\bottomrule
\end{tabular}

\begin{tabular}{l}
\toprule
Minisymposium presentation\\
(Multiscale models of infectious diseases)\\
\bottomrule
\end{tabular}
\vskip0.5cm

  Mathematical modelling is an important tool for understanding and controlling the spread of infectious diseases. Heterogeneity, which refers to differences in factors such as demographics, behaviour, susceptibility, infectiousness, and disease severity within a population, plays a critical role in disease transmission and control. Incorporating heterogeneity into models can help researchers better understand disease spread across subpopulations and design more targeted control strategies. However, heterogeneous models can be high-dimensional and complex, leading to theoretical challenges in modelling analysis. Moreover, data collections in the field are often in the forms of aggregation, making modeling implementation challenging. In this talk, we will discuss recent developments and remaining challenges in modeling infectious diseases with a focus on heterogeneity and aggregation. The goal is to provide attendees with valuable insights into the significance of incorporating heterogeneity into models and effective ways to address associated challenges. 


\vskip\medskipamount

  \leaders\vrule width \textwidth\vskip0.4pt

  \vskip\medskipamount

  \nointerlineskip

  \pagebreak[2]

\end{absolutelynopagebreak}

\begin{absolutelynopagebreak}
\subsection{Morgan Craig - Delays in the cell cycle: implications in immune responses} 

\begin{tabular}{l}
\toprule
Sainte-Justine University Hospital Research Centre / Université de Montréal\\
Canada\\
\bottomrule
\end{tabular}

\begin{tabular}{l}
\toprule
Minisymposium presentation\\
(Delay-differential equations in applications)\\
\bottomrule
\end{tabular}
\vskip0.5cm

  Mathematical modelling is an important tool for understanding and controlling the spread of infectious diseases. Heterogeneity, which refers to differences in factors such as demographics, behaviour, susceptibility, infectiousness, and disease severity within a population, plays a critical role in disease transmission and control. Incorporating heterogeneity into models can help researchers better understand disease spread across subpopulations and design more targeted control strategies. However, heterogeneous models can be high-dimensional and complex, leading to theoretical challenges in modelling analysis. Moreover, data collections in the field are often in the forms of aggregation, making modeling implementation challenging. In this talk, we will discuss recent developments and remaining challenges in modeling infectious diseases with a focus on heterogeneity and aggregation. The goal is to provide attendees with valuable insights into the significance of incorporating heterogeneity into models and effective ways to address associated challenges. 


\vskip\medskipamount

  \leaders\vrule width \textwidth\vskip0.4pt

  \vskip\medskipamount

  \nointerlineskip

  \pagebreak[2]

\end{absolutelynopagebreak}

\begin{absolutelynopagebreak}
\subsection{Morgan Craig - The TME determines the efficacy of immunotherapies to treat glioblastoma} 

\begin{tabular}{l}
\toprule
Sainte-Justine University Hospital Research Centre / Université de Montréal\\
Canada\\
\bottomrule
\end{tabular}

\begin{tabular}{l}
\toprule
Minisymposium presentation\\
(Modelling the Cancer Microenvironment)\\
\bottomrule
\end{tabular}
\vskip0.5cm

  Mathematical modelling is an important tool for understanding and controlling the spread of infectious diseases. Heterogeneity, which refers to differences in factors such as demographics, behaviour, susceptibility, infectiousness, and disease severity within a population, plays a critical role in disease transmission and control. Incorporating heterogeneity into models can help researchers better understand disease spread across subpopulations and design more targeted control strategies. However, heterogeneous models can be high-dimensional and complex, leading to theoretical challenges in modelling analysis. Moreover, data collections in the field are often in the forms of aggregation, making modeling implementation challenging. In this talk, we will discuss recent developments and remaining challenges in modeling infectious diseases with a focus on heterogeneity and aggregation. The goal is to provide attendees with valuable insights into the significance of incorporating heterogeneity into models and effective ways to address associated challenges. 


\vskip\medskipamount

  \leaders\vrule width \textwidth\vskip0.4pt

  \vskip\medskipamount

  \nointerlineskip

  \pagebreak[2]

\end{absolutelynopagebreak}

\begin{absolutelynopagebreak}
\subsection{Jim Cushing - Discrete-time models of infectious diseases: a project in memory of Aziz-Abdul Yakubu} 

\begin{tabular}{l}
\toprule
University of Arizona\\
USA\\
\bottomrule
\end{tabular}

\begin{tabular}{l}
\toprule
Plenary presentation\\
\bottomrule
\end{tabular}
\vskip0.5cm

  Mathematical modelling is an important tool for understanding and controlling the spread of infectious diseases. Heterogeneity, which refers to differences in factors such as demographics, behaviour, susceptibility, infectiousness, and disease severity within a population, plays a critical role in disease transmission and control. Incorporating heterogeneity into models can help researchers better understand disease spread across subpopulations and design more targeted control strategies. However, heterogeneous models can be high-dimensional and complex, leading to theoretical challenges in modelling analysis. Moreover, data collections in the field are often in the forms of aggregation, making modeling implementation challenging. In this talk, we will discuss recent developments and remaining challenges in modeling infectious diseases with a focus on heterogeneity and aggregation. The goal is to provide attendees with valuable insights into the significance of incorporating heterogeneity into models and effective ways to address associated challenges. 


\vskip\medskipamount

  \leaders\vrule width \textwidth\vskip0.4pt

  \vskip\medskipamount

  \nointerlineskip

  \pagebreak[2]

\end{absolutelynopagebreak}

\begin{absolutelynopagebreak}
\subsection{Tanuja Das - An eclipse-phase lag drives oscillations in a viral infection model with a general growth function} 

\begin{tabular}{l}
\toprule
University of New Brunswick, New Brunswick\\
Canada\\
\bottomrule
\end{tabular}

\begin{tabular}{l}
\toprule
Minisymposium presentation\\
(Recent Advances in Modelling Infectious Diseases)\\
\bottomrule
\end{tabular}
\vskip0.5cm

  Mathematical modelling is an important tool for understanding and controlling the spread of infectious diseases. Heterogeneity, which refers to differences in factors such as demographics, behaviour, susceptibility, infectiousness, and disease severity within a population, plays a critical role in disease transmission and control. Incorporating heterogeneity into models can help researchers better understand disease spread across subpopulations and design more targeted control strategies. However, heterogeneous models can be high-dimensional and complex, leading to theoretical challenges in modelling analysis. Moreover, data collections in the field are often in the forms of aggregation, making modeling implementation challenging. In this talk, we will discuss recent developments and remaining challenges in modeling infectious diseases with a focus on heterogeneity and aggregation. The goal is to provide attendees with valuable insights into the significance of incorporating heterogeneity into models and effective ways to address associated challenges. 


\vskip\medskipamount

  \leaders\vrule width \textwidth\vskip0.4pt

  \vskip\medskipamount

  \nointerlineskip

  \pagebreak[2]

\end{absolutelynopagebreak}

\begin{absolutelynopagebreak}
\subsection{Tanuja Das - Effect of a novel generalized incidence rate function in SIR model: stability switches and bifurcations} 

\begin{tabular}{l}
\toprule
University of New Brunswick, New Brunswick\\
Canada\\
\bottomrule
\end{tabular}

\begin{tabular}{l}
\toprule
Contributed presentation\\
\bottomrule
\end{tabular}
\vskip0.5cm

  Mathematical modelling is an important tool for understanding and controlling the spread of infectious diseases. Heterogeneity, which refers to differences in factors such as demographics, behaviour, susceptibility, infectiousness, and disease severity within a population, plays a critical role in disease transmission and control. Incorporating heterogeneity into models can help researchers better understand disease spread across subpopulations and design more targeted control strategies. However, heterogeneous models can be high-dimensional and complex, leading to theoretical challenges in modelling analysis. Moreover, data collections in the field are often in the forms of aggregation, making modeling implementation challenging. In this talk, we will discuss recent developments and remaining challenges in modeling infectious diseases with a focus on heterogeneity and aggregation. The goal is to provide attendees with valuable insights into the significance of incorporating heterogeneity into models and effective ways to address associated challenges. 


\vskip\medskipamount

  \leaders\vrule width \textwidth\vskip0.4pt

  \vskip\medskipamount

  \nointerlineskip

  \pagebreak[2]

\end{absolutelynopagebreak}

\begin{absolutelynopagebreak}
\subsection{Xiaoyan Deng - Predicting heterogeneous CD8+ immune memory responses in COVID-19 using a virtual patient cohort} 

\begin{tabular}{l}
\toprule
Université de Montréal\\
Canada\\
\bottomrule
\end{tabular}

\begin{tabular}{l}
\toprule
Minisymposium presentation\\
(Within-host and between-host mathematical models of biological dynamics)\\
\bottomrule
\end{tabular}
\vskip0.5cm

  Mathematical modelling is an important tool for understanding and controlling the spread of infectious diseases. Heterogeneity, which refers to differences in factors such as demographics, behaviour, susceptibility, infectiousness, and disease severity within a population, plays a critical role in disease transmission and control. Incorporating heterogeneity into models can help researchers better understand disease spread across subpopulations and design more targeted control strategies. However, heterogeneous models can be high-dimensional and complex, leading to theoretical challenges in modelling analysis. Moreover, data collections in the field are often in the forms of aggregation, making modeling implementation challenging. In this talk, we will discuss recent developments and remaining challenges in modeling infectious diseases with a focus on heterogeneity and aggregation. The goal is to provide attendees with valuable insights into the significance of incorporating heterogeneity into models and effective ways to address associated challenges. 


\vskip\medskipamount

  \leaders\vrule width \textwidth\vskip0.4pt

  \vskip\medskipamount

  \nointerlineskip

  \pagebreak[2]

\end{absolutelynopagebreak}

\begin{absolutelynopagebreak}
\subsection{Clotilde Djuikem - Impulsive modelling of rust dynamics and predator releases} 

\begin{tabular}{l}
\toprule
Université Côte d’Azur, Inria, INRAE, CNRS, Université Paris Sorbonne, BIOCORE, France\\
France\\
\bottomrule
\end{tabular}

\begin{tabular}{l}
\toprule
Contributed presentation\\
\bottomrule
\end{tabular}
\vskip0.5cm

  Mathematical modelling is an important tool for understanding and controlling the spread of infectious diseases. Heterogeneity, which refers to differences in factors such as demographics, behaviour, susceptibility, infectiousness, and disease severity within a population, plays a critical role in disease transmission and control. Incorporating heterogeneity into models can help researchers better understand disease spread across subpopulations and design more targeted control strategies. However, heterogeneous models can be high-dimensional and complex, leading to theoretical challenges in modelling analysis. Moreover, data collections in the field are often in the forms of aggregation, making modeling implementation challenging. In this talk, we will discuss recent developments and remaining challenges in modeling infectious diseases with a focus on heterogeneity and aggregation. The goal is to provide attendees with valuable insights into the significance of incorporating heterogeneity into models and effective ways to address associated challenges. 


\vskip\medskipamount

  \leaders\vrule width \textwidth\vskip0.4pt

  \vskip\medskipamount

  \nointerlineskip

  \pagebreak[2]

\end{absolutelynopagebreak}

\begin{absolutelynopagebreak}
\subsection{Marisa Eisenberg - Models to inform wastewater-based epidemiology: identifiability, uncertainty, and opportunities} 

\begin{tabular}{l}
\toprule
University of Michigan, Ann Arbor\\
USA\\
\bottomrule
\end{tabular}

\begin{tabular}{l}
\toprule
Minisymposium presentation\\
(Recent Advances in Modelling Infectious Diseases)\\
\bottomrule
\end{tabular}
\vskip0.5cm

  Mathematical modelling is an important tool for understanding and controlling the spread of infectious diseases. Heterogeneity, which refers to differences in factors such as demographics, behaviour, susceptibility, infectiousness, and disease severity within a population, plays a critical role in disease transmission and control. Incorporating heterogeneity into models can help researchers better understand disease spread across subpopulations and design more targeted control strategies. However, heterogeneous models can be high-dimensional and complex, leading to theoretical challenges in modelling analysis. Moreover, data collections in the field are often in the forms of aggregation, making modeling implementation challenging. In this talk, we will discuss recent developments and remaining challenges in modeling infectious diseases with a focus on heterogeneity and aggregation. The goal is to provide attendees with valuable insights into the significance of incorporating heterogeneity into models and effective ways to address associated challenges. 


\vskip\medskipamount

  \leaders\vrule width \textwidth\vskip0.4pt

  \vskip\medskipamount

  \nointerlineskip

  \pagebreak[2]

\end{absolutelynopagebreak}

\begin{absolutelynopagebreak}
\subsection{Marisa Eisenberg - Identifiability and infectious disease interventions: exploring when uncertainty matters} 

\begin{tabular}{l}
\toprule
University of Michigan, Ann Arbor\\
USA\\
\bottomrule
\end{tabular}

\begin{tabular}{l}
\toprule
Plenary presentation\\
\bottomrule
\end{tabular}
\vskip0.5cm

  Mathematical modelling is an important tool for understanding and controlling the spread of infectious diseases. Heterogeneity, which refers to differences in factors such as demographics, behaviour, susceptibility, infectiousness, and disease severity within a population, plays a critical role in disease transmission and control. Incorporating heterogeneity into models can help researchers better understand disease spread across subpopulations and design more targeted control strategies. However, heterogeneous models can be high-dimensional and complex, leading to theoretical challenges in modelling analysis. Moreover, data collections in the field are often in the forms of aggregation, making modeling implementation challenging. In this talk, we will discuss recent developments and remaining challenges in modeling infectious diseases with a focus on heterogeneity and aggregation. The goal is to provide attendees with valuable insights into the significance of incorporating heterogeneity into models and effective ways to address associated challenges. 


\vskip\medskipamount

  \leaders\vrule width \textwidth\vskip0.4pt

  \vskip\medskipamount

  \nointerlineskip

  \pagebreak[2]

\end{absolutelynopagebreak}

\begin{absolutelynopagebreak}
\subsection{Blessing Emerenini - Data Assimilation of Quorum Sensing Regulation of Bacteria-Phage Interaction in Biofilm} 

\begin{tabular}{l}
\toprule
Rochester Institute of Technology\\
USA\\
\bottomrule
\end{tabular}

\begin{tabular}{l}
\toprule
Minisymposium presentation\\
(Within-host and between-host mathematical models of biological dynamics)\\
\bottomrule
\end{tabular}
\vskip0.5cm

  Mathematical modelling is an important tool for understanding and controlling the spread of infectious diseases. Heterogeneity, which refers to differences in factors such as demographics, behaviour, susceptibility, infectiousness, and disease severity within a population, plays a critical role in disease transmission and control. Incorporating heterogeneity into models can help researchers better understand disease spread across subpopulations and design more targeted control strategies. However, heterogeneous models can be high-dimensional and complex, leading to theoretical challenges in modelling analysis. Moreover, data collections in the field are often in the forms of aggregation, making modeling implementation challenging. In this talk, we will discuss recent developments and remaining challenges in modeling infectious diseases with a focus on heterogeneity and aggregation. The goal is to provide attendees with valuable insights into the significance of incorporating heterogeneity into models and effective ways to address associated challenges. 


\vskip\medskipamount

  \leaders\vrule width \textwidth\vskip0.4pt

  \vskip\medskipamount

  \nointerlineskip

  \pagebreak[2]

\end{absolutelynopagebreak}

\begin{absolutelynopagebreak}
\subsection{Guihong Fan - Delayed model for the transmission and control of COVID-19 with Fangcang Shelter Hospitals} 

\begin{tabular}{l}
\toprule
Columbus State University\\
USA\\
\bottomrule
\end{tabular}

\begin{tabular}{l}
\toprule
Minisymposium presentation\\
(Delay-differential equations in applications)\\
\bottomrule
\end{tabular}
\vskip0.5cm

  Mathematical modelling is an important tool for understanding and controlling the spread of infectious diseases. Heterogeneity, which refers to differences in factors such as demographics, behaviour, susceptibility, infectiousness, and disease severity within a population, plays a critical role in disease transmission and control. Incorporating heterogeneity into models can help researchers better understand disease spread across subpopulations and design more targeted control strategies. However, heterogeneous models can be high-dimensional and complex, leading to theoretical challenges in modelling analysis. Moreover, data collections in the field are often in the forms of aggregation, making modeling implementation challenging. In this talk, we will discuss recent developments and remaining challenges in modeling infectious diseases with a focus on heterogeneity and aggregation. The goal is to provide attendees with valuable insights into the significance of incorporating heterogeneity into models and effective ways to address associated challenges. 


\vskip\medskipamount

  \leaders\vrule width \textwidth\vskip0.4pt

  \vskip\medskipamount

  \nointerlineskip

  \pagebreak[2]

\end{absolutelynopagebreak}

\begin{absolutelynopagebreak}
\subsection{Suzan Farhang-Sardroodi - Mathematical model of muscle wasting in cancer cachexia incorporated with immunology} 

\begin{tabular}{l}
\toprule
Department of Mathematics, university of Manitoba\\
Canada\\
\bottomrule
\end{tabular}

\begin{tabular}{l}
\toprule
Minisymposium presentation\\
(Mathematical and computational approaches to modelling immunology)\\
\bottomrule
\end{tabular}
\vskip0.5cm

  Mathematical modelling is an important tool for understanding and controlling the spread of infectious diseases. Heterogeneity, which refers to differences in factors such as demographics, behaviour, susceptibility, infectiousness, and disease severity within a population, plays a critical role in disease transmission and control. Incorporating heterogeneity into models can help researchers better understand disease spread across subpopulations and design more targeted control strategies. However, heterogeneous models can be high-dimensional and complex, leading to theoretical challenges in modelling analysis. Moreover, data collections in the field are often in the forms of aggregation, making modeling implementation challenging. In this talk, we will discuss recent developments and remaining challenges in modeling infectious diseases with a focus on heterogeneity and aggregation. The goal is to provide attendees with valuable insights into the significance of incorporating heterogeneity into models and effective ways to address associated challenges. 


\vskip\medskipamount

  \leaders\vrule width \textwidth\vskip0.4pt

  \vskip\medskipamount

  \nointerlineskip

  \pagebreak[2]

\end{absolutelynopagebreak}

\begin{absolutelynopagebreak}
\subsection{Suzan Farhang-Sardroodi - Mathematical Modelling of the Impact of Human Immune Diversity on COVID-19 transmission} 

\begin{tabular}{l}
\toprule
Department of Mathematics, university of Manitoba\\
Canada\\
\bottomrule
\end{tabular}

\begin{tabular}{l}
\toprule
Minisymposium presentation\\
(Multiscale models of infectious diseases)\\
\bottomrule
\end{tabular}
\vskip0.5cm

  Mathematical modelling is an important tool for understanding and controlling the spread of infectious diseases. Heterogeneity, which refers to differences in factors such as demographics, behaviour, susceptibility, infectiousness, and disease severity within a population, plays a critical role in disease transmission and control. Incorporating heterogeneity into models can help researchers better understand disease spread across subpopulations and design more targeted control strategies. However, heterogeneous models can be high-dimensional and complex, leading to theoretical challenges in modelling analysis. Moreover, data collections in the field are often in the forms of aggregation, making modeling implementation challenging. In this talk, we will discuss recent developments and remaining challenges in modeling infectious diseases with a focus on heterogeneity and aggregation. The goal is to provide attendees with valuable insights into the significance of incorporating heterogeneity into models and effective ways to address associated challenges. 


\vskip\medskipamount

  \leaders\vrule width \textwidth\vskip0.4pt

  \vskip\medskipamount

  \nointerlineskip

  \pagebreak[2]

\end{absolutelynopagebreak}

\begin{absolutelynopagebreak}
\subsection{Jonathan Forde - Modeling the challenges of optimal resource deployment for epidemic prevention} 

\begin{tabular}{l}
\toprule
Hobart and William Smith Colleges\\
USA\\
\bottomrule
\end{tabular}

\begin{tabular}{l}
\toprule
Minisymposium presentation\\
(Multiscale models of infectious diseases)\\
\bottomrule
\end{tabular}
\vskip0.5cm

  Mathematical modelling is an important tool for understanding and controlling the spread of infectious diseases. Heterogeneity, which refers to differences in factors such as demographics, behaviour, susceptibility, infectiousness, and disease severity within a population, plays a critical role in disease transmission and control. Incorporating heterogeneity into models can help researchers better understand disease spread across subpopulations and design more targeted control strategies. However, heterogeneous models can be high-dimensional and complex, leading to theoretical challenges in modelling analysis. Moreover, data collections in the field are often in the forms of aggregation, making modeling implementation challenging. In this talk, we will discuss recent developments and remaining challenges in modeling infectious diseases with a focus on heterogeneity and aggregation. The goal is to provide attendees with valuable insights into the significance of incorporating heterogeneity into models and effective ways to address associated challenges. 


\vskip\medskipamount

  \leaders\vrule width \textwidth\vskip0.4pt

  \vskip\medskipamount

  \nointerlineskip

  \pagebreak[2]

\end{absolutelynopagebreak}

\begin{absolutelynopagebreak}
\subsection{Samaneh Gholami - Mathematical Modeling of Immune Response to Protein Subunit COVID-19 Vaccines} 

\begin{tabular}{l}
\toprule
York University\\
Canada\\
\bottomrule
\end{tabular}

\begin{tabular}{l}
\toprule
Minisymposium presentation\\
(Within-host and between-host mathematical models of biological dynamics)\\
\bottomrule
\end{tabular}
\vskip0.5cm

  Mathematical modelling is an important tool for understanding and controlling the spread of infectious diseases. Heterogeneity, which refers to differences in factors such as demographics, behaviour, susceptibility, infectiousness, and disease severity within a population, plays a critical role in disease transmission and control. Incorporating heterogeneity into models can help researchers better understand disease spread across subpopulations and design more targeted control strategies. However, heterogeneous models can be high-dimensional and complex, leading to theoretical challenges in modelling analysis. Moreover, data collections in the field are often in the forms of aggregation, making modeling implementation challenging. In this talk, we will discuss recent developments and remaining challenges in modeling infectious diseases with a focus on heterogeneity and aggregation. The goal is to provide attendees with valuable insights into the significance of incorporating heterogeneity into models and effective ways to address associated challenges. 


\vskip\medskipamount

  \leaders\vrule width \textwidth\vskip0.4pt

  \vskip\medskipamount

  \nointerlineskip

  \pagebreak[2]

\end{absolutelynopagebreak}

\begin{absolutelynopagebreak}
\subsection{Abba Gumel - Mathematical Assessment of the Role of Pre-Exposure Prophylaxis on the HIV Pandemic} 

\begin{tabular}{l}
\toprule
University of Maryland\\
USA\\
\bottomrule
\end{tabular}

\begin{tabular}{l}
\toprule
Minisymposium presentation\\
(Recent Advances in Modelling Infectious Diseases)\\
\bottomrule
\end{tabular}
\vskip0.5cm

  Mathematical modelling is an important tool for understanding and controlling the spread of infectious diseases. Heterogeneity, which refers to differences in factors such as demographics, behaviour, susceptibility, infectiousness, and disease severity within a population, plays a critical role in disease transmission and control. Incorporating heterogeneity into models can help researchers better understand disease spread across subpopulations and design more targeted control strategies. However, heterogeneous models can be high-dimensional and complex, leading to theoretical challenges in modelling analysis. Moreover, data collections in the field are often in the forms of aggregation, making modeling implementation challenging. In this talk, we will discuss recent developments and remaining challenges in modeling infectious diseases with a focus on heterogeneity and aggregation. The goal is to provide attendees with valuable insights into the significance of incorporating heterogeneity into models and effective ways to address associated challenges. 


\vskip\medskipamount

  \leaders\vrule width \textwidth\vskip0.4pt

  \vskip\medskipamount

  \nointerlineskip

  \pagebreak[2]

\end{absolutelynopagebreak}

\begin{absolutelynopagebreak}
\subsection{Abba Gumel - Mathematics of Wolbachia-based biocontrol of mosquito-borne diseases} 

\begin{tabular}{l}
\toprule
University of Maryland\\
USA\\
\bottomrule
\end{tabular}

\begin{tabular}{l}
\toprule
Minisymposium presentation\\
(Vector-Borne Disease Dynamics)\\
\bottomrule
\end{tabular}
\vskip0.5cm

  Mathematical modelling is an important tool for understanding and controlling the spread of infectious diseases. Heterogeneity, which refers to differences in factors such as demographics, behaviour, susceptibility, infectiousness, and disease severity within a population, plays a critical role in disease transmission and control. Incorporating heterogeneity into models can help researchers better understand disease spread across subpopulations and design more targeted control strategies. However, heterogeneous models can be high-dimensional and complex, leading to theoretical challenges in modelling analysis. Moreover, data collections in the field are often in the forms of aggregation, making modeling implementation challenging. In this talk, we will discuss recent developments and remaining challenges in modeling infectious diseases with a focus on heterogeneity and aggregation. The goal is to provide attendees with valuable insights into the significance of incorporating heterogeneity into models and effective ways to address associated challenges. 


\vskip\medskipamount

  \leaders\vrule width \textwidth\vskip0.4pt

  \vskip\medskipamount

  \nointerlineskip

  \pagebreak[2]

\end{absolutelynopagebreak}

\begin{absolutelynopagebreak}
\subsection{Donglin Han - Retrospective estimation of proportion of total infections of COVID-19 during the first wave in Alberta} 

\begin{tabular}{l}
\toprule
University of Alberta\\
Canada\\
\bottomrule
\end{tabular}

\begin{tabular}{l}
\toprule
Minisymposium presentation\\
(Recent Advances in Modelling Infectious Diseases)\\
\bottomrule
\end{tabular}
\vskip0.5cm

  Mathematical modelling is an important tool for understanding and controlling the spread of infectious diseases. Heterogeneity, which refers to differences in factors such as demographics, behaviour, susceptibility, infectiousness, and disease severity within a population, plays a critical role in disease transmission and control. Incorporating heterogeneity into models can help researchers better understand disease spread across subpopulations and design more targeted control strategies. However, heterogeneous models can be high-dimensional and complex, leading to theoretical challenges in modelling analysis. Moreover, data collections in the field are often in the forms of aggregation, making modeling implementation challenging. In this talk, we will discuss recent developments and remaining challenges in modeling infectious diseases with a focus on heterogeneity and aggregation. The goal is to provide attendees with valuable insights into the significance of incorporating heterogeneity into models and effective ways to address associated challenges. 


\vskip\medskipamount

  \leaders\vrule width \textwidth\vskip0.4pt

  \vskip\medskipamount

  \nointerlineskip

  \pagebreak[2]

\end{absolutelynopagebreak}

\begin{absolutelynopagebreak}
\subsection{Katsuma Hayashi - Reconstructing the temporal dynamics of clustering from cluster surveillance of COVID-19} 

\begin{tabular}{l}
\toprule
Kyoto University\\
Japan\\
\bottomrule
\end{tabular}

\begin{tabular}{l}
\toprule
Minisymposium presentation\\
(Real time epidemiology in various geographic scales)\\
\bottomrule
\end{tabular}
\vskip0.5cm

  Mathematical modelling is an important tool for understanding and controlling the spread of infectious diseases. Heterogeneity, which refers to differences in factors such as demographics, behaviour, susceptibility, infectiousness, and disease severity within a population, plays a critical role in disease transmission and control. Incorporating heterogeneity into models can help researchers better understand disease spread across subpopulations and design more targeted control strategies. However, heterogeneous models can be high-dimensional and complex, leading to theoretical challenges in modelling analysis. Moreover, data collections in the field are often in the forms of aggregation, making modeling implementation challenging. In this talk, we will discuss recent developments and remaining challenges in modeling infectious diseases with a focus on heterogeneity and aggregation. The goal is to provide attendees with valuable insights into the significance of incorporating heterogeneity into models and effective ways to address associated challenges. 


\vskip\medskipamount

  \leaders\vrule width \textwidth\vskip0.4pt

  \vskip\medskipamount

  \nointerlineskip

  \pagebreak[2]

\end{absolutelynopagebreak}

\begin{absolutelynopagebreak}
\subsection{Jane Heffernan - Seasonality and Influenza pH1N12009 Vaccination Impact} 

\begin{tabular}{l}
\toprule
York University\\
Canada\\
\bottomrule
\end{tabular}

\begin{tabular}{l}
\toprule
Minisymposium presentation\\
(Recent Advances in Modelling Infectious Diseases)\\
\bottomrule
\end{tabular}
\vskip0.5cm

  Mathematical modelling is an important tool for understanding and controlling the spread of infectious diseases. Heterogeneity, which refers to differences in factors such as demographics, behaviour, susceptibility, infectiousness, and disease severity within a population, plays a critical role in disease transmission and control. Incorporating heterogeneity into models can help researchers better understand disease spread across subpopulations and design more targeted control strategies. However, heterogeneous models can be high-dimensional and complex, leading to theoretical challenges in modelling analysis. Moreover, data collections in the field are often in the forms of aggregation, making modeling implementation challenging. In this talk, we will discuss recent developments and remaining challenges in modeling infectious diseases with a focus on heterogeneity and aggregation. The goal is to provide attendees with valuable insights into the significance of incorporating heterogeneity into models and effective ways to address associated challenges. 


\vskip\medskipamount

  \leaders\vrule width \textwidth\vskip0.4pt

  \vskip\medskipamount

  \nointerlineskip

  \pagebreak[2]

\end{absolutelynopagebreak}

\begin{absolutelynopagebreak}
\subsection{Jane Heffernan - Modelling Immunity to SARS-CoV-2} 

\begin{tabular}{l}
\toprule
York University\\
Canada\\
\bottomrule
\end{tabular}

\begin{tabular}{l}
\toprule
Plenary presentation\\
\bottomrule
\end{tabular}
\vskip0.5cm

  Mathematical modelling is an important tool for understanding and controlling the spread of infectious diseases. Heterogeneity, which refers to differences in factors such as demographics, behaviour, susceptibility, infectiousness, and disease severity within a population, plays a critical role in disease transmission and control. Incorporating heterogeneity into models can help researchers better understand disease spread across subpopulations and design more targeted control strategies. However, heterogeneous models can be high-dimensional and complex, leading to theoretical challenges in modelling analysis. Moreover, data collections in the field are often in the forms of aggregation, making modeling implementation challenging. In this talk, we will discuss recent developments and remaining challenges in modeling infectious diseases with a focus on heterogeneity and aggregation. The goal is to provide attendees with valuable insights into the significance of incorporating heterogeneity into models and effective ways to address associated challenges. 


\vskip\medskipamount

  \leaders\vrule width \textwidth\vskip0.4pt

  \vskip\medskipamount

  \nointerlineskip

  \pagebreak[2]

\end{absolutelynopagebreak}

\begin{absolutelynopagebreak}
\subsection{Esteban A. Hernandez-Vargas - The Shapes of Immunological Data during Respiratory Infections} 

\begin{tabular}{l}
\toprule
University of Idaho\\
USA\\
\bottomrule
\end{tabular}

\begin{tabular}{l}
\toprule
Plenary presentation\\
\bottomrule
\end{tabular}
\vskip0.5cm

  Mathematical modelling is an important tool for understanding and controlling the spread of infectious diseases. Heterogeneity, which refers to differences in factors such as demographics, behaviour, susceptibility, infectiousness, and disease severity within a population, plays a critical role in disease transmission and control. Incorporating heterogeneity into models can help researchers better understand disease spread across subpopulations and design more targeted control strategies. However, heterogeneous models can be high-dimensional and complex, leading to theoretical challenges in modelling analysis. Moreover, data collections in the field are often in the forms of aggregation, making modeling implementation challenging. In this talk, we will discuss recent developments and remaining challenges in modeling infectious diseases with a focus on heterogeneity and aggregation. The goal is to provide attendees with valuable insights into the significance of incorporating heterogeneity into models and effective ways to address associated challenges. 


\vskip\medskipamount

  \leaders\vrule width \textwidth\vskip0.4pt

  \vskip\medskipamount

  \nointerlineskip

  \pagebreak[2]

\end{absolutelynopagebreak}

\begin{absolutelynopagebreak}
\subsection{Sarafa Iyaniwura - Understanding the efficacy of capsid protein allosteric modulators using a multiscale model of hepatitis B virus} 

\begin{tabular}{l}
\toprule
Los Alamos National Laboratory\\
USA\\
\bottomrule
\end{tabular}

\begin{tabular}{l}
\toprule
Minisymposium presentation\\
(Within-host and between-host mathematical models of biological dynamics)\\
\bottomrule
\end{tabular}
\vskip0.5cm

  Mathematical modelling is an important tool for understanding and controlling the spread of infectious diseases. Heterogeneity, which refers to differences in factors such as demographics, behaviour, susceptibility, infectiousness, and disease severity within a population, plays a critical role in disease transmission and control. Incorporating heterogeneity into models can help researchers better understand disease spread across subpopulations and design more targeted control strategies. However, heterogeneous models can be high-dimensional and complex, leading to theoretical challenges in modelling analysis. Moreover, data collections in the field are often in the forms of aggregation, making modeling implementation challenging. In this talk, we will discuss recent developments and remaining challenges in modeling infectious diseases with a focus on heterogeneity and aggregation. The goal is to provide attendees with valuable insights into the significance of incorporating heterogeneity into models and effective ways to address associated challenges. 


\vskip\medskipamount

  \leaders\vrule width \textwidth\vskip0.4pt

  \vskip\medskipamount

  \nointerlineskip

  \pagebreak[2]

\end{absolutelynopagebreak}

\begin{absolutelynopagebreak}
\subsection{Sana Jahedi - Addressing Waning Immunity Against Measles: Reevaluating the MMR Vaccination Program} 

\begin{tabular}{l}
\toprule
McMaster University, Biology department\\
Canada\\
\bottomrule
\end{tabular}

\begin{tabular}{l}
\toprule
Minisymposium presentation\\
(Mathematical and computational approaches to modelling immunology)\\
\bottomrule
\end{tabular}
\vskip0.5cm

  Mathematical modelling is an important tool for understanding and controlling the spread of infectious diseases. Heterogeneity, which refers to differences in factors such as demographics, behaviour, susceptibility, infectiousness, and disease severity within a population, plays a critical role in disease transmission and control. Incorporating heterogeneity into models can help researchers better understand disease spread across subpopulations and design more targeted control strategies. However, heterogeneous models can be high-dimensional and complex, leading to theoretical challenges in modelling analysis. Moreover, data collections in the field are often in the forms of aggregation, making modeling implementation challenging. In this talk, we will discuss recent developments and remaining challenges in modeling infectious diseases with a focus on heterogeneity and aggregation. The goal is to provide attendees with valuable insights into the significance of incorporating heterogeneity into models and effective ways to address associated challenges. 


\vskip\medskipamount

  \leaders\vrule width \textwidth\vskip0.4pt

  \vskip\medskipamount

  \nointerlineskip

  \pagebreak[2]

\end{absolutelynopagebreak}

\begin{absolutelynopagebreak}
\subsection{Harsh Vardhan Jain - A quantitative evaluation of an anti-cancer vaccine for treating advanced prostate cancer} 

\begin{tabular}{l}
\toprule
Department of Mathematics and Statistics, University of Minnesota Duluth\\
USA\\
\bottomrule
\end{tabular}

\begin{tabular}{l}
\toprule
Minisymposium presentation\\
(Mathematical modeling and analysis in cancer immunotherapy)\\
\bottomrule
\end{tabular}
\vskip0.5cm

  Mathematical modelling is an important tool for understanding and controlling the spread of infectious diseases. Heterogeneity, which refers to differences in factors such as demographics, behaviour, susceptibility, infectiousness, and disease severity within a population, plays a critical role in disease transmission and control. Incorporating heterogeneity into models can help researchers better understand disease spread across subpopulations and design more targeted control strategies. However, heterogeneous models can be high-dimensional and complex, leading to theoretical challenges in modelling analysis. Moreover, data collections in the field are often in the forms of aggregation, making modeling implementation challenging. In this talk, we will discuss recent developments and remaining challenges in modeling infectious diseases with a focus on heterogeneity and aggregation. The goal is to provide attendees with valuable insights into the significance of incorporating heterogeneity into models and effective ways to address associated challenges. 


\vskip\medskipamount

  \leaders\vrule width \textwidth\vskip0.4pt

  \vskip\medskipamount

  \nointerlineskip

  \pagebreak[2]

\end{absolutelynopagebreak}

\begin{absolutelynopagebreak}
\subsection{Marek Kimmel - Site frequency spectra and estimation of clonal dynamics of tumors} 

\begin{tabular}{l}
\toprule
Departments of Statistics and Bioengineering, Rice University\\
USA\\
\bottomrule
\end{tabular}

\begin{tabular}{l}
\toprule
Minisymposium presentation\\
(Stochastic population models: Theory and applications in Cancer Research)\\
\bottomrule
\end{tabular}
\vskip0.5cm

  Mathematical modelling is an important tool for understanding and controlling the spread of infectious diseases. Heterogeneity, which refers to differences in factors such as demographics, behaviour, susceptibility, infectiousness, and disease severity within a population, plays a critical role in disease transmission and control. Incorporating heterogeneity into models can help researchers better understand disease spread across subpopulations and design more targeted control strategies. However, heterogeneous models can be high-dimensional and complex, leading to theoretical challenges in modelling analysis. Moreover, data collections in the field are often in the forms of aggregation, making modeling implementation challenging. In this talk, we will discuss recent developments and remaining challenges in modeling infectious diseases with a focus on heterogeneity and aggregation. The goal is to provide attendees with valuable insights into the significance of incorporating heterogeneity into models and effective ways to address associated challenges. 


\vskip\medskipamount

  \leaders\vrule width \textwidth\vskip0.4pt

  \vskip\medskipamount

  \nointerlineskip

  \pagebreak[2]

\end{absolutelynopagebreak}

\begin{absolutelynopagebreak}
\subsection{Jude Kong - Mpox dynamic model: incorporating adaptive behavioural changes, different control strategies in the MSM community \& under-reporting} 

\begin{tabular}{l}
\toprule
York University\\
Canada\\
\bottomrule
\end{tabular}

\begin{tabular}{l}
\toprule
Minisymposium presentation\\
(Recent Advances in Modelling Infectious Diseases)\\
\bottomrule
\end{tabular}
\vskip0.5cm

  Mathematical modelling is an important tool for understanding and controlling the spread of infectious diseases. Heterogeneity, which refers to differences in factors such as demographics, behaviour, susceptibility, infectiousness, and disease severity within a population, plays a critical role in disease transmission and control. Incorporating heterogeneity into models can help researchers better understand disease spread across subpopulations and design more targeted control strategies. However, heterogeneous models can be high-dimensional and complex, leading to theoretical challenges in modelling analysis. Moreover, data collections in the field are often in the forms of aggregation, making modeling implementation challenging. In this talk, we will discuss recent developments and remaining challenges in modeling infectious diseases with a focus on heterogeneity and aggregation. The goal is to provide attendees with valuable insights into the significance of incorporating heterogeneity into models and effective ways to address associated challenges. 


\vskip\medskipamount

  \leaders\vrule width \textwidth\vskip0.4pt

  \vskip\medskipamount

  \nointerlineskip

  \pagebreak[2]

\end{absolutelynopagebreak}

\begin{absolutelynopagebreak}
\subsection{Jude Kong - Leveraging  mathematical models to support early management of an emerging disease outbreak: the case of Covid-19 and Africa} 

\begin{tabular}{l}
\toprule
York University\\
Canada\\
\bottomrule
\end{tabular}

\begin{tabular}{l}
\toprule
Minisymposium presentation\\
(Within-host and between-host mathematical models of biological dynamics)\\
\bottomrule
\end{tabular}
\vskip0.5cm

  Mathematical modelling is an important tool for understanding and controlling the spread of infectious diseases. Heterogeneity, which refers to differences in factors such as demographics, behaviour, susceptibility, infectiousness, and disease severity within a population, plays a critical role in disease transmission and control. Incorporating heterogeneity into models can help researchers better understand disease spread across subpopulations and design more targeted control strategies. However, heterogeneous models can be high-dimensional and complex, leading to theoretical challenges in modelling analysis. Moreover, data collections in the field are often in the forms of aggregation, making modeling implementation challenging. In this talk, we will discuss recent developments and remaining challenges in modeling infectious diseases with a focus on heterogeneity and aggregation. The goal is to provide attendees with valuable insights into the significance of incorporating heterogeneity into models and effective ways to address associated challenges. 


\vskip\medskipamount

  \leaders\vrule width \textwidth\vskip0.4pt

  \vskip\medskipamount

  \nointerlineskip

  \pagebreak[2]

\end{absolutelynopagebreak}

\begin{absolutelynopagebreak}
\subsection{Chapin Korosec - Longitudinal immunological outcomes from three doses of COVID-19 vaccines in people living with HIV: antibodies, memory-B cells, cytokines, and a novel within-host immunological model} 

\begin{tabular}{l}
\toprule
York University\\
Canada\\
\bottomrule
\end{tabular}

\begin{tabular}{l}
\toprule
Minisymposium presentation\\
(Mathematical and computational approaches to modelling immunology)\\
\bottomrule
\end{tabular}
\vskip0.5cm

  Mathematical modelling is an important tool for understanding and controlling the spread of infectious diseases. Heterogeneity, which refers to differences in factors such as demographics, behaviour, susceptibility, infectiousness, and disease severity within a population, plays a critical role in disease transmission and control. Incorporating heterogeneity into models can help researchers better understand disease spread across subpopulations and design more targeted control strategies. However, heterogeneous models can be high-dimensional and complex, leading to theoretical challenges in modelling analysis. Moreover, data collections in the field are often in the forms of aggregation, making modeling implementation challenging. In this talk, we will discuss recent developments and remaining challenges in modeling infectious diseases with a focus on heterogeneity and aggregation. The goal is to provide attendees with valuable insights into the significance of incorporating heterogeneity into models and effective ways to address associated challenges. 


\vskip\medskipamount

  \leaders\vrule width \textwidth\vskip0.4pt

  \vskip\medskipamount

  \nointerlineskip

  \pagebreak[2]

\end{absolutelynopagebreak}

\begin{absolutelynopagebreak}
\subsection{Christopher Kribs - Impact of tetravalent dengue vaccination with screening, ADE, and altered infectivity on dengue and Zika transmission} 

\begin{tabular}{l}
\toprule
University of Texas at Arlington\\
USA\\
\bottomrule
\end{tabular}

\begin{tabular}{l}
\toprule
Minisymposium presentation\\
(Vector-Borne Disease Dynamics)\\
\bottomrule
\end{tabular}
\vskip0.5cm

  Mathematical modelling is an important tool for understanding and controlling the spread of infectious diseases. Heterogeneity, which refers to differences in factors such as demographics, behaviour, susceptibility, infectiousness, and disease severity within a population, plays a critical role in disease transmission and control. Incorporating heterogeneity into models can help researchers better understand disease spread across subpopulations and design more targeted control strategies. However, heterogeneous models can be high-dimensional and complex, leading to theoretical challenges in modelling analysis. Moreover, data collections in the field are often in the forms of aggregation, making modeling implementation challenging. In this talk, we will discuss recent developments and remaining challenges in modeling infectious diseases with a focus on heterogeneity and aggregation. The goal is to provide attendees with valuable insights into the significance of incorporating heterogeneity into models and effective ways to address associated challenges. 


\vskip\medskipamount

  \leaders\vrule width \textwidth\vskip0.4pt

  \vskip\medskipamount

  \nointerlineskip

  \pagebreak[2]

\end{absolutelynopagebreak}

\begin{absolutelynopagebreak}
\subsection{Furkan Kurtoglu - Modeling Colorectal Cancer Spheroids using Agent-Based Modeling Including Metabolism} 

\begin{tabular}{l}
\toprule
Indiana University\\
USA\\
\bottomrule
\end{tabular}

\begin{tabular}{l}
\toprule
Minisymposium presentation\\
(Modelling the Cancer Microenvironment)\\
\bottomrule
\end{tabular}
\vskip0.5cm

  Mathematical modelling is an important tool for understanding and controlling the spread of infectious diseases. Heterogeneity, which refers to differences in factors such as demographics, behaviour, susceptibility, infectiousness, and disease severity within a population, plays a critical role in disease transmission and control. Incorporating heterogeneity into models can help researchers better understand disease spread across subpopulations and design more targeted control strategies. However, heterogeneous models can be high-dimensional and complex, leading to theoretical challenges in modelling analysis. Moreover, data collections in the field are often in the forms of aggregation, making modeling implementation challenging. In this talk, we will discuss recent developments and remaining challenges in modeling infectious diseases with a focus on heterogeneity and aggregation. The goal is to provide attendees with valuable insights into the significance of incorporating heterogeneity into models and effective ways to address associated challenges. 


\vskip\medskipamount

  \leaders\vrule width \textwidth\vskip0.4pt

  \vskip\medskipamount

  \nointerlineskip

  \pagebreak[2]

\end{absolutelynopagebreak}

\begin{absolutelynopagebreak}
\subsection{Brandon Legried - Inferring phylogenetic birth-death models from extant lineages through time} 

\begin{tabular}{l}
\toprule
Georgia Institute of Technology - School of Mathematics\\
USA\\
\bottomrule
\end{tabular}

\begin{tabular}{l}
\toprule
Minisymposium presentation\\
(Stochastic population models: Theory and applications in Cancer Research)\\
\bottomrule
\end{tabular}
\vskip0.5cm

  Mathematical modelling is an important tool for understanding and controlling the spread of infectious diseases. Heterogeneity, which refers to differences in factors such as demographics, behaviour, susceptibility, infectiousness, and disease severity within a population, plays a critical role in disease transmission and control. Incorporating heterogeneity into models can help researchers better understand disease spread across subpopulations and design more targeted control strategies. However, heterogeneous models can be high-dimensional and complex, leading to theoretical challenges in modelling analysis. Moreover, data collections in the field are often in the forms of aggregation, making modeling implementation challenging. In this talk, we will discuss recent developments and remaining challenges in modeling infectious diseases with a focus on heterogeneity and aggregation. The goal is to provide attendees with valuable insights into the significance of incorporating heterogeneity into models and effective ways to address associated challenges. 


\vskip\medskipamount

  \leaders\vrule width \textwidth\vskip0.4pt

  \vskip\medskipamount

  \nointerlineskip

  \pagebreak[2]

\end{absolutelynopagebreak}

\begin{absolutelynopagebreak}
\subsection{Kang-Ling Liao - The opposite functions and treatment outcomes of CD200-CD200R in cancer} 

\begin{tabular}{l}
\toprule
Mathematics, University of Manitoba\\
Canada\\
\bottomrule
\end{tabular}

\begin{tabular}{l}
\toprule
Minisymposium presentation\\
(Mathematical modeling and analysis in cancer immunotherapy)\\
\bottomrule
\end{tabular}
\vskip0.5cm

  Mathematical modelling is an important tool for understanding and controlling the spread of infectious diseases. Heterogeneity, which refers to differences in factors such as demographics, behaviour, susceptibility, infectiousness, and disease severity within a population, plays a critical role in disease transmission and control. Incorporating heterogeneity into models can help researchers better understand disease spread across subpopulations and design more targeted control strategies. However, heterogeneous models can be high-dimensional and complex, leading to theoretical challenges in modelling analysis. Moreover, data collections in the field are often in the forms of aggregation, making modeling implementation challenging. In this talk, we will discuss recent developments and remaining challenges in modeling infectious diseases with a focus on heterogeneity and aggregation. The goal is to provide attendees with valuable insights into the significance of incorporating heterogeneity into models and effective ways to address associated challenges. 


\vskip\medskipamount

  \leaders\vrule width \textwidth\vskip0.4pt

  \vskip\medskipamount

  \nointerlineskip

  \pagebreak[2]

\end{absolutelynopagebreak}

\begin{absolutelynopagebreak}
\subsection{Ernesto Lima - Development and calibration of a stochastic, multiscale agent-based model for predicting tumor and vasculature growth} 

\begin{tabular}{l}
\toprule
The University of Texas at Austin\\
USA\\
\bottomrule
\end{tabular}

\begin{tabular}{l}
\toprule
Minisymposium presentation\\
(Modelling the Cancer Microenvironment)\\
\bottomrule
\end{tabular}
\vskip0.5cm

  Mathematical modelling is an important tool for understanding and controlling the spread of infectious diseases. Heterogeneity, which refers to differences in factors such as demographics, behaviour, susceptibility, infectiousness, and disease severity within a population, plays a critical role in disease transmission and control. Incorporating heterogeneity into models can help researchers better understand disease spread across subpopulations and design more targeted control strategies. However, heterogeneous models can be high-dimensional and complex, leading to theoretical challenges in modelling analysis. Moreover, data collections in the field are often in the forms of aggregation, making modeling implementation challenging. In this talk, we will discuss recent developments and remaining challenges in modeling infectious diseases with a focus on heterogeneity and aggregation. The goal is to provide attendees with valuable insights into the significance of incorporating heterogeneity into models and effective ways to address associated challenges. 


\vskip\medskipamount

  \leaders\vrule width \textwidth\vskip0.4pt

  \vskip\medskipamount

  \nointerlineskip

  \pagebreak[2]

\end{absolutelynopagebreak}

\begin{absolutelynopagebreak}
\subsection{Xiaochen Long - A Branching Process Model of Clonal Hematopoiesis} 

\begin{tabular}{l}
\toprule
Rice University\\
USA\\
\bottomrule
\end{tabular}

\begin{tabular}{l}
\toprule
Minisymposium presentation\\
(Stochastic population models: Theory and applications in Cancer Research)\\
\bottomrule
\end{tabular}
\vskip0.5cm

  Mathematical modelling is an important tool for understanding and controlling the spread of infectious diseases. Heterogeneity, which refers to differences in factors such as demographics, behaviour, susceptibility, infectiousness, and disease severity within a population, plays a critical role in disease transmission and control. Incorporating heterogeneity into models can help researchers better understand disease spread across subpopulations and design more targeted control strategies. However, heterogeneous models can be high-dimensional and complex, leading to theoretical challenges in modelling analysis. Moreover, data collections in the field are often in the forms of aggregation, making modeling implementation challenging. In this talk, we will discuss recent developments and remaining challenges in modeling infectious diseases with a focus on heterogeneity and aggregation. The goal is to provide attendees with valuable insights into the significance of incorporating heterogeneity into models and effective ways to address associated challenges. 


\vskip\medskipamount

  \leaders\vrule width \textwidth\vskip0.4pt

  \vskip\medskipamount

  \nointerlineskip

  \pagebreak[2]

\end{absolutelynopagebreak}

\begin{absolutelynopagebreak}
\subsection{Loïc Louison - A Population Harvesting Model with Time and size Competition Dependence Function} 

\begin{tabular}{l}
\toprule
Université de Guyane\\
France\\
\bottomrule
\end{tabular}

\begin{tabular}{l}
\toprule
Contributed presentation\\
\bottomrule
\end{tabular}
\vskip0.5cm

  Mathematical modelling is an important tool for understanding and controlling the spread of infectious diseases. Heterogeneity, which refers to differences in factors such as demographics, behaviour, susceptibility, infectiousness, and disease severity within a population, plays a critical role in disease transmission and control. Incorporating heterogeneity into models can help researchers better understand disease spread across subpopulations and design more targeted control strategies. However, heterogeneous models can be high-dimensional and complex, leading to theoretical challenges in modelling analysis. Moreover, data collections in the field are often in the forms of aggregation, making modeling implementation challenging. In this talk, we will discuss recent developments and remaining challenges in modeling infectious diseases with a focus on heterogeneity and aggregation. The goal is to provide attendees with valuable insights into the significance of incorporating heterogeneity into models and effective ways to address associated challenges. 


\vskip\medskipamount

  \leaders\vrule width \textwidth\vskip0.4pt

  \vskip\medskipamount

  \nointerlineskip

  \pagebreak[2]

\end{absolutelynopagebreak}

\begin{absolutelynopagebreak}
\subsection{Nadia Loy - A non-local kinetic model for cell migration : a study of the interplay between contact guidance and steric hindrance} 

\begin{tabular}{l}
\toprule
Politecnico di Torino\\
Italy\\
\bottomrule
\end{tabular}

\begin{tabular}{l}
\toprule
Minisymposium presentation\\
(Modelling the Cancer Microenvironment)\\
\bottomrule
\end{tabular}
\vskip0.5cm

  Mathematical modelling is an important tool for understanding and controlling the spread of infectious diseases. Heterogeneity, which refers to differences in factors such as demographics, behaviour, susceptibility, infectiousness, and disease severity within a population, plays a critical role in disease transmission and control. Incorporating heterogeneity into models can help researchers better understand disease spread across subpopulations and design more targeted control strategies. However, heterogeneous models can be high-dimensional and complex, leading to theoretical challenges in modelling analysis. Moreover, data collections in the field are often in the forms of aggregation, making modeling implementation challenging. In this talk, we will discuss recent developments and remaining challenges in modeling infectious diseases with a focus on heterogeneity and aggregation. The goal is to provide attendees with valuable insights into the significance of incorporating heterogeneity into models and effective ways to address associated challenges. 


\vskip\medskipamount

  \leaders\vrule width \textwidth\vskip0.4pt

  \vskip\medskipamount

  \nointerlineskip

  \pagebreak[2]

\end{absolutelynopagebreak}

\begin{absolutelynopagebreak}
\subsection{Chinwendu Emilian Madubueze - Modelling transmission dynamics of Lassa fever transmission with two environmental pathway transmissions} 

\begin{tabular}{l}
\toprule
York university Toronto, Ontario\\
Canada\\
\bottomrule
\end{tabular}

\begin{tabular}{l}
\toprule
Contributed presentation\\
\bottomrule
\end{tabular}
\vskip0.5cm

  Mathematical modelling is an important tool for understanding and controlling the spread of infectious diseases. Heterogeneity, which refers to differences in factors such as demographics, behaviour, susceptibility, infectiousness, and disease severity within a population, plays a critical role in disease transmission and control. Incorporating heterogeneity into models can help researchers better understand disease spread across subpopulations and design more targeted control strategies. However, heterogeneous models can be high-dimensional and complex, leading to theoretical challenges in modelling analysis. Moreover, data collections in the field are often in the forms of aggregation, making modeling implementation challenging. In this talk, we will discuss recent developments and remaining challenges in modeling infectious diseases with a focus on heterogeneity and aggregation. The goal is to provide attendees with valuable insights into the significance of incorporating heterogeneity into models and effective ways to address associated challenges. 


\vskip\medskipamount

  \leaders\vrule width \textwidth\vskip0.4pt

  \vskip\medskipamount

  \nointerlineskip

  \pagebreak[2]

\end{absolutelynopagebreak}

\begin{absolutelynopagebreak}
\subsection{Anna Marciniak-Czochra - Evolution of stem cell populations:  Mechanistic mathematical modelling vs single cell data} 

\begin{tabular}{l}
\toprule
Heidelberg University\\
Germany\\
\bottomrule
\end{tabular}

\begin{tabular}{l}
\toprule
Plenary presentation\\
\bottomrule
\end{tabular}
\vskip0.5cm

  Mathematical modelling is an important tool for understanding and controlling the spread of infectious diseases. Heterogeneity, which refers to differences in factors such as demographics, behaviour, susceptibility, infectiousness, and disease severity within a population, plays a critical role in disease transmission and control. Incorporating heterogeneity into models can help researchers better understand disease spread across subpopulations and design more targeted control strategies. However, heterogeneous models can be high-dimensional and complex, leading to theoretical challenges in modelling analysis. Moreover, data collections in the field are often in the forms of aggregation, making modeling implementation challenging. In this talk, we will discuss recent developments and remaining challenges in modeling infectious diseases with a focus on heterogeneity and aggregation. The goal is to provide attendees with valuable insights into the significance of incorporating heterogeneity into models and effective ways to address associated challenges. 


\vskip\medskipamount

  \leaders\vrule width \textwidth\vskip0.4pt

  \vskip\medskipamount

  \nointerlineskip

  \pagebreak[2]

\end{absolutelynopagebreak}

\begin{absolutelynopagebreak}
\subsection{Fabio Milner - A mosquito-bird-human model for West Nile virus disease transmission} 

\begin{tabular}{l}
\toprule
Arizona State University\\
USA\\
\bottomrule
\end{tabular}

\begin{tabular}{l}
\toprule
Minisymposium presentation\\
(Vector-Borne Disease Dynamics)\\
\bottomrule
\end{tabular}
\vskip0.5cm

  Mathematical modelling is an important tool for understanding and controlling the spread of infectious diseases. Heterogeneity, which refers to differences in factors such as demographics, behaviour, susceptibility, infectiousness, and disease severity within a population, plays a critical role in disease transmission and control. Incorporating heterogeneity into models can help researchers better understand disease spread across subpopulations and design more targeted control strategies. However, heterogeneous models can be high-dimensional and complex, leading to theoretical challenges in modelling analysis. Moreover, data collections in the field are often in the forms of aggregation, making modeling implementation challenging. In this talk, we will discuss recent developments and remaining challenges in modeling infectious diseases with a focus on heterogeneity and aggregation. The goal is to provide attendees with valuable insights into the significance of incorporating heterogeneity into models and effective ways to address associated challenges. 


\vskip\medskipamount

  \leaders\vrule width \textwidth\vskip0.4pt

  \vskip\medskipamount

  \nointerlineskip

  \pagebreak[2]

\end{absolutelynopagebreak}

\begin{absolutelynopagebreak}
\subsection{Jemal Mohammed-Awel - Mathematics model for assessing the impacts of pyrethroid resistance and temperature on population abundance of malaria mosquitoes} 

\begin{tabular}{l}
\toprule
Morgan State University\\
USA\\
\bottomrule
\end{tabular}

\begin{tabular}{l}
\toprule
Minisymposium presentation\\
(Vector-Borne Disease Dynamics)\\
\bottomrule
\end{tabular}
\vskip0.5cm

  Mathematical modelling is an important tool for understanding and controlling the spread of infectious diseases. Heterogeneity, which refers to differences in factors such as demographics, behaviour, susceptibility, infectiousness, and disease severity within a population, plays a critical role in disease transmission and control. Incorporating heterogeneity into models can help researchers better understand disease spread across subpopulations and design more targeted control strategies. However, heterogeneous models can be high-dimensional and complex, leading to theoretical challenges in modelling analysis. Moreover, data collections in the field are often in the forms of aggregation, making modeling implementation challenging. In this talk, we will discuss recent developments and remaining challenges in modeling infectious diseases with a focus on heterogeneity and aggregation. The goal is to provide attendees with valuable insights into the significance of incorporating heterogeneity into models and effective ways to address associated challenges. 


\vskip\medskipamount

  \leaders\vrule width \textwidth\vskip0.4pt

  \vskip\medskipamount

  \nointerlineskip

  \pagebreak[2]

\end{absolutelynopagebreak}

\begin{absolutelynopagebreak}
\subsection{Nicola Mulberry - A nested model for pneumococcal population dynamics} 

\begin{tabular}{l}
\toprule
Simon Fraser University\\
Canada\\
\bottomrule
\end{tabular}

\begin{tabular}{l}
\toprule
Minisymposium presentation\\
(Bridging the scale from within-host to epidemic models)\\
\bottomrule
\end{tabular}
\vskip0.5cm

  Mathematical modelling is an important tool for understanding and controlling the spread of infectious diseases. Heterogeneity, which refers to differences in factors such as demographics, behaviour, susceptibility, infectiousness, and disease severity within a population, plays a critical role in disease transmission and control. Incorporating heterogeneity into models can help researchers better understand disease spread across subpopulations and design more targeted control strategies. However, heterogeneous models can be high-dimensional and complex, leading to theoretical challenges in modelling analysis. Moreover, data collections in the field are often in the forms of aggregation, making modeling implementation challenging. In this talk, we will discuss recent developments and remaining challenges in modeling infectious diseases with a focus on heterogeneity and aggregation. The goal is to provide attendees with valuable insights into the significance of incorporating heterogeneity into models and effective ways to address associated challenges. 


\vskip\medskipamount

  \leaders\vrule width \textwidth\vskip0.4pt

  \vskip\medskipamount

  \nointerlineskip

  \pagebreak[2]

\end{absolutelynopagebreak}

\begin{absolutelynopagebreak}
\subsection{Toshiyuki Namba - Unexpected coexistence and extinction in an intraguild predation system} 

\begin{tabular}{l}
\toprule
Osaka Metropolitan University\\
Japan\\
\bottomrule
\end{tabular}

\begin{tabular}{l}
\toprule
Contributed presentation\\
\bottomrule
\end{tabular}
\vskip0.5cm

  Mathematical modelling is an important tool for understanding and controlling the spread of infectious diseases. Heterogeneity, which refers to differences in factors such as demographics, behaviour, susceptibility, infectiousness, and disease severity within a population, plays a critical role in disease transmission and control. Incorporating heterogeneity into models can help researchers better understand disease spread across subpopulations and design more targeted control strategies. However, heterogeneous models can be high-dimensional and complex, leading to theoretical challenges in modelling analysis. Moreover, data collections in the field are often in the forms of aggregation, making modeling implementation challenging. In this talk, we will discuss recent developments and remaining challenges in modeling infectious diseases with a focus on heterogeneity and aggregation. The goal is to provide attendees with valuable insights into the significance of incorporating heterogeneity into models and effective ways to address associated challenges. 


\vskip\medskipamount

  \leaders\vrule width \textwidth\vskip0.4pt

  \vskip\medskipamount

  \nointerlineskip

  \pagebreak[2]

\end{absolutelynopagebreak}

\begin{absolutelynopagebreak}
\subsection{Jay Newby - Dynamic self organization and microscale fluid properties of nucleoplasm} 

\begin{tabular}{l}
\toprule
University of Alberta\\
Canada\\
\bottomrule
\end{tabular}

\begin{tabular}{l}
\toprule
Plenary presentation\\
\bottomrule
\end{tabular}
\vskip0.5cm

  Mathematical modelling is an important tool for understanding and controlling the spread of infectious diseases. Heterogeneity, which refers to differences in factors such as demographics, behaviour, susceptibility, infectiousness, and disease severity within a population, plays a critical role in disease transmission and control. Incorporating heterogeneity into models can help researchers better understand disease spread across subpopulations and design more targeted control strategies. However, heterogeneous models can be high-dimensional and complex, leading to theoretical challenges in modelling analysis. Moreover, data collections in the field are often in the forms of aggregation, making modeling implementation challenging. In this talk, we will discuss recent developments and remaining challenges in modeling infectious diseases with a focus on heterogeneity and aggregation. The goal is to provide attendees with valuable insights into the significance of incorporating heterogeneity into models and effective ways to address associated challenges. 


\vskip\medskipamount

  \leaders\vrule width \textwidth\vskip0.4pt

  \vskip\medskipamount

  \nointerlineskip

  \pagebreak[2]

\end{absolutelynopagebreak}

\begin{absolutelynopagebreak}
\subsection{Hiroshi Nishiura - Night-time population consistently explains the transmission dynamics of coronavirus disease 2019 in three megacities in Japan} 

\begin{tabular}{l}
\toprule
Kyoto University\\
Japan\\
\bottomrule
\end{tabular}

\begin{tabular}{l}
\toprule
Minisymposium presentation\\
(Real time epidemiology in various geographic scales)\\
\bottomrule
\end{tabular}
\vskip0.5cm

  Mathematical modelling is an important tool for understanding and controlling the spread of infectious diseases. Heterogeneity, which refers to differences in factors such as demographics, behaviour, susceptibility, infectiousness, and disease severity within a population, plays a critical role in disease transmission and control. Incorporating heterogeneity into models can help researchers better understand disease spread across subpopulations and design more targeted control strategies. However, heterogeneous models can be high-dimensional and complex, leading to theoretical challenges in modelling analysis. Moreover, data collections in the field are often in the forms of aggregation, making modeling implementation challenging. In this talk, we will discuss recent developments and remaining challenges in modeling infectious diseases with a focus on heterogeneity and aggregation. The goal is to provide attendees with valuable insights into the significance of incorporating heterogeneity into models and effective ways to address associated challenges. 


\vskip\medskipamount

  \leaders\vrule width \textwidth\vskip0.4pt

  \vskip\medskipamount

  \nointerlineskip

  \pagebreak[2]

\end{absolutelynopagebreak}

\begin{absolutelynopagebreak}
\subsection{Ryo Oizumi - Analytical Representation of Eigensystem in Multiregional Leslie Matrix Model: Application to Sensitivity Analysis of Population Declining in Japan} 

\begin{tabular}{l}
\toprule
National Institute of Population and Social Security Research\\
Japan\\
\bottomrule
\end{tabular}

\begin{tabular}{l}
\toprule
Contributed presentation\\
\bottomrule
\end{tabular}
\vskip0.5cm

  Mathematical modelling is an important tool for understanding and controlling the spread of infectious diseases. Heterogeneity, which refers to differences in factors such as demographics, behaviour, susceptibility, infectiousness, and disease severity within a population, plays a critical role in disease transmission and control. Incorporating heterogeneity into models can help researchers better understand disease spread across subpopulations and design more targeted control strategies. However, heterogeneous models can be high-dimensional and complex, leading to theoretical challenges in modelling analysis. Moreover, data collections in the field are often in the forms of aggregation, making modeling implementation challenging. In this talk, we will discuss recent developments and remaining challenges in modeling infectious diseases with a focus on heterogeneity and aggregation. The goal is to provide attendees with valuable insights into the significance of incorporating heterogeneity into models and effective ways to address associated challenges. 


\vskip\medskipamount

  \leaders\vrule width \textwidth\vskip0.4pt

  \vskip\medskipamount

  \nointerlineskip

  \pagebreak[2]

\end{absolutelynopagebreak}

\begin{absolutelynopagebreak}
\subsection{Lorenzo Pellis - Multi-scale time-since-infection models in evolutionary epidemiology} 

\begin{tabular}{l}
\toprule
The University of Manchester\\
UK\\
\bottomrule
\end{tabular}

\begin{tabular}{l}
\toprule
Minisymposium presentation\\
(Bridging the scale from within-host to epidemic models)\\
\bottomrule
\end{tabular}
\vskip0.5cm

  Mathematical modelling is an important tool for understanding and controlling the spread of infectious diseases. Heterogeneity, which refers to differences in factors such as demographics, behaviour, susceptibility, infectiousness, and disease severity within a population, plays a critical role in disease transmission and control. Incorporating heterogeneity into models can help researchers better understand disease spread across subpopulations and design more targeted control strategies. However, heterogeneous models can be high-dimensional and complex, leading to theoretical challenges in modelling analysis. Moreover, data collections in the field are often in the forms of aggregation, making modeling implementation challenging. In this talk, we will discuss recent developments and remaining challenges in modeling infectious diseases with a focus on heterogeneity and aggregation. The goal is to provide attendees with valuable insights into the significance of incorporating heterogeneity into models and effective ways to address associated challenges. 


\vskip\medskipamount

  \leaders\vrule width \textwidth\vskip0.4pt

  \vskip\medskipamount

  \nointerlineskip

  \pagebreak[2]

\end{absolutelynopagebreak}

\begin{absolutelynopagebreak}
\subsection{Tin Phan - Modeling the emergence of viral resistance in SARS-CoV-2 patients treated with an anti-spike monoclonal antibody} 

\begin{tabular}{l}
\toprule
Los Alamos National Laboratory\\
USA\\
\bottomrule
\end{tabular}

\begin{tabular}{l}
\toprule
Minisymposium presentation\\
(Mathematical and computational approaches to modelling immunology)\\
\bottomrule
\end{tabular}
\vskip0.5cm

  Mathematical modelling is an important tool for understanding and controlling the spread of infectious diseases. Heterogeneity, which refers to differences in factors such as demographics, behaviour, susceptibility, infectiousness, and disease severity within a population, plays a critical role in disease transmission and control. Incorporating heterogeneity into models can help researchers better understand disease spread across subpopulations and design more targeted control strategies. However, heterogeneous models can be high-dimensional and complex, leading to theoretical challenges in modelling analysis. Moreover, data collections in the field are often in the forms of aggregation, making modeling implementation challenging. In this talk, we will discuss recent developments and remaining challenges in modeling infectious diseases with a focus on heterogeneity and aggregation. The goal is to provide attendees with valuable insights into the significance of incorporating heterogeneity into models and effective ways to address associated challenges. 


\vskip\medskipamount

  \leaders\vrule width \textwidth\vskip0.4pt

  \vskip\medskipamount

  \nointerlineskip

  \pagebreak[2]

\end{absolutelynopagebreak}

\begin{absolutelynopagebreak}
\subsection{Tin Phan - Integrating wastewater surveillance data with epidemic models: challenges and opportunities} 

\begin{tabular}{l}
\toprule
Los Alamos National Laboratory\\
USA\\
\bottomrule
\end{tabular}

\begin{tabular}{l}
\toprule
Minisymposium presentation\\
(Multiscale models of infectious diseases)\\
\bottomrule
\end{tabular}
\vskip0.5cm

  Mathematical modelling is an important tool for understanding and controlling the spread of infectious diseases. Heterogeneity, which refers to differences in factors such as demographics, behaviour, susceptibility, infectiousness, and disease severity within a population, plays a critical role in disease transmission and control. Incorporating heterogeneity into models can help researchers better understand disease spread across subpopulations and design more targeted control strategies. However, heterogeneous models can be high-dimensional and complex, leading to theoretical challenges in modelling analysis. Moreover, data collections in the field are often in the forms of aggregation, making modeling implementation challenging. In this talk, we will discuss recent developments and remaining challenges in modeling infectious diseases with a focus on heterogeneity and aggregation. The goal is to provide attendees with valuable insights into the significance of incorporating heterogeneity into models and effective ways to address associated challenges. 


\vskip\medskipamount

  \leaders\vrule width \textwidth\vskip0.4pt

  \vskip\medskipamount

  \nointerlineskip

  \pagebreak[2]

\end{absolutelynopagebreak}

\begin{absolutelynopagebreak}
\subsection{Tanya Philippsen - A retrospective modelling analysis of the effect of control measures on the transmission of SARS-CoV-2 in Canada} 

\begin{tabular}{l}
\toprule
University of Victoria\\
Canada\\
\bottomrule
\end{tabular}

\begin{tabular}{l}
\toprule
Minisymposium presentation\\
(Recent Advances in Modelling Infectious Diseases)\\
\bottomrule
\end{tabular}
\vskip0.5cm

  Mathematical modelling is an important tool for understanding and controlling the spread of infectious diseases. Heterogeneity, which refers to differences in factors such as demographics, behaviour, susceptibility, infectiousness, and disease severity within a population, plays a critical role in disease transmission and control. Incorporating heterogeneity into models can help researchers better understand disease spread across subpopulations and design more targeted control strategies. However, heterogeneous models can be high-dimensional and complex, leading to theoretical challenges in modelling analysis. Moreover, data collections in the field are often in the forms of aggregation, making modeling implementation challenging. In this talk, we will discuss recent developments and remaining challenges in modeling infectious diseases with a focus on heterogeneity and aggregation. The goal is to provide attendees with valuable insights into the significance of incorporating heterogeneity into models and effective ways to address associated challenges. 


\vskip\medskipamount

  \leaders\vrule width \textwidth\vskip0.4pt

  \vskip\medskipamount

  \nointerlineskip

  \pagebreak[2]

\end{absolutelynopagebreak}

\begin{absolutelynopagebreak}
\subsection{Andrea Pugliese - Combining data from surveillance on mosquitoes and corvids to understand the factors affecting the dynamics of West Nile Virus in Emilia-Romagna, Italy} 

\begin{tabular}{l}
\toprule
Dept. of Mathematics, University of Trento\\
Italy\\
\bottomrule
\end{tabular}

\begin{tabular}{l}
\toprule
Minisymposium presentation\\
(Vector-Borne Disease Dynamics)\\
\bottomrule
\end{tabular}
\vskip0.5cm

  Mathematical modelling is an important tool for understanding and controlling the spread of infectious diseases. Heterogeneity, which refers to differences in factors such as demographics, behaviour, susceptibility, infectiousness, and disease severity within a population, plays a critical role in disease transmission and control. Incorporating heterogeneity into models can help researchers better understand disease spread across subpopulations and design more targeted control strategies. However, heterogeneous models can be high-dimensional and complex, leading to theoretical challenges in modelling analysis. Moreover, data collections in the field are often in the forms of aggregation, making modeling implementation challenging. In this talk, we will discuss recent developments and remaining challenges in modeling infectious diseases with a focus on heterogeneity and aggregation. The goal is to provide attendees with valuable insights into the significance of incorporating heterogeneity into models and effective ways to address associated challenges. 


\vskip\medskipamount

  \leaders\vrule width \textwidth\vskip0.4pt

  \vskip\medskipamount

  \nointerlineskip

  \pagebreak[2]

\end{absolutelynopagebreak}

\begin{absolutelynopagebreak}
\subsection{Erica Rutter - Modeling and Estimating Intratumoral Heterogeneity in Cancer} 

\begin{tabular}{l}
\toprule
University of California, Merced\\
USA\\
\bottomrule
\end{tabular}

\begin{tabular}{l}
\toprule
Minisymposium presentation\\
(Modelling the Cancer Microenvironment)\\
\bottomrule
\end{tabular}
\vskip0.5cm

  Mathematical modelling is an important tool for understanding and controlling the spread of infectious diseases. Heterogeneity, which refers to differences in factors such as demographics, behaviour, susceptibility, infectiousness, and disease severity within a population, plays a critical role in disease transmission and control. Incorporating heterogeneity into models can help researchers better understand disease spread across subpopulations and design more targeted control strategies. However, heterogeneous models can be high-dimensional and complex, leading to theoretical challenges in modelling analysis. Moreover, data collections in the field are often in the forms of aggregation, making modeling implementation challenging. In this talk, we will discuss recent developments and remaining challenges in modeling infectious diseases with a focus on heterogeneity and aggregation. The goal is to provide attendees with valuable insights into the significance of incorporating heterogeneity into models and effective ways to address associated challenges. 


\vskip\medskipamount

  \leaders\vrule width \textwidth\vskip0.4pt

  \vskip\medskipamount

  \nointerlineskip

  \pagebreak[2]

\end{absolutelynopagebreak}

\begin{absolutelynopagebreak}
\subsection{Erica Rutter - Global Sensitivity Analysis of a Structured Model of COVID-19 Transmission on a College Campus} 

\begin{tabular}{l}
\toprule
University of California, Merced\\
USA\\
\bottomrule
\end{tabular}

\begin{tabular}{l}
\toprule
Minisymposium presentation\\
(Multiscale models of infectious diseases)\\
\bottomrule
\end{tabular}
\vskip0.5cm

  Mathematical modelling is an important tool for understanding and controlling the spread of infectious diseases. Heterogeneity, which refers to differences in factors such as demographics, behaviour, susceptibility, infectiousness, and disease severity within a population, plays a critical role in disease transmission and control. Incorporating heterogeneity into models can help researchers better understand disease spread across subpopulations and design more targeted control strategies. However, heterogeneous models can be high-dimensional and complex, leading to theoretical challenges in modelling analysis. Moreover, data collections in the field are often in the forms of aggregation, making modeling implementation challenging. In this talk, we will discuss recent developments and remaining challenges in modeling infectious diseases with a focus on heterogeneity and aggregation. The goal is to provide attendees with valuable insights into the significance of incorporating heterogeneity into models and effective ways to address associated challenges. 


\vskip\medskipamount

  \leaders\vrule width \textwidth\vskip0.4pt

  \vskip\medskipamount

  \nointerlineskip

  \pagebreak[2]

\end{absolutelynopagebreak}

\begin{absolutelynopagebreak}
\subsection{Paul Salceanu - Robust uniform persistence for structured models of delay differential equations} 

\begin{tabular}{l}
\toprule
University of Louisiana at Lafayette\\
USA\\
\bottomrule
\end{tabular}

\begin{tabular}{l}
\toprule
Minisymposium presentation\\
(Ecological and Epidemiological Models with Dispersal)\\
\bottomrule
\end{tabular}
\vskip0.5cm

  Mathematical modelling is an important tool for understanding and controlling the spread of infectious diseases. Heterogeneity, which refers to differences in factors such as demographics, behaviour, susceptibility, infectiousness, and disease severity within a population, plays a critical role in disease transmission and control. Incorporating heterogeneity into models can help researchers better understand disease spread across subpopulations and design more targeted control strategies. However, heterogeneous models can be high-dimensional and complex, leading to theoretical challenges in modelling analysis. Moreover, data collections in the field are often in the forms of aggregation, making modeling implementation challenging. In this talk, we will discuss recent developments and remaining challenges in modeling infectious diseases with a focus on heterogeneity and aggregation. The goal is to provide attendees with valuable insights into the significance of incorporating heterogeneity into models and effective ways to address associated challenges. 


\vskip\medskipamount

  \leaders\vrule width \textwidth\vskip0.4pt

  \vskip\medskipamount

  \nointerlineskip

  \pagebreak[2]

\end{absolutelynopagebreak}

\begin{absolutelynopagebreak}
\subsection{Leili Shahriyari - Digital twins of cancer patients: a step toward personalized treatments} 

\begin{tabular}{l}
\toprule
Department of Mathematics \& Statistics, University of Massachusetts Amherst\\
USA\\
\bottomrule
\end{tabular}

\begin{tabular}{l}
\toprule
Minisymposium presentation\\
(Mathematical modeling and analysis in cancer immunotherapy)\\
\bottomrule
\end{tabular}
\vskip0.5cm

  Mathematical modelling is an important tool for understanding and controlling the spread of infectious diseases. Heterogeneity, which refers to differences in factors such as demographics, behaviour, susceptibility, infectiousness, and disease severity within a population, plays a critical role in disease transmission and control. Incorporating heterogeneity into models can help researchers better understand disease spread across subpopulations and design more targeted control strategies. However, heterogeneous models can be high-dimensional and complex, leading to theoretical challenges in modelling analysis. Moreover, data collections in the field are often in the forms of aggregation, making modeling implementation challenging. In this talk, we will discuss recent developments and remaining challenges in modeling infectious diseases with a focus on heterogeneity and aggregation. The goal is to provide attendees with valuable insights into the significance of incorporating heterogeneity into models and effective ways to address associated challenges. 


\vskip\medskipamount

  \leaders\vrule width \textwidth\vskip0.4pt

  \vskip\medskipamount

  \nointerlineskip

  \pagebreak[2]

\end{absolutelynopagebreak}

\begin{absolutelynopagebreak}
\subsection{Zhisheng Shuai - Heterogeneity and Aggregation in Modeling Infectious Diseases} 

\begin{tabular}{l}
\toprule
University of Central Florida\\
USA\\
\bottomrule
\end{tabular}

\begin{tabular}{l}
\toprule
Plenary presentation\\
\bottomrule
\end{tabular}
\vskip0.5cm

  Mathematical modelling is an important tool for understanding and controlling the spread of infectious diseases. Heterogeneity, which refers to differences in factors such as demographics, behaviour, susceptibility, infectiousness, and disease severity within a population, plays a critical role in disease transmission and control. Incorporating heterogeneity into models can help researchers better understand disease spread across subpopulations and design more targeted control strategies. However, heterogeneous models can be high-dimensional and complex, leading to theoretical challenges in modelling analysis. Moreover, data collections in the field are often in the forms of aggregation, making modeling implementation challenging. In this talk, we will discuss recent developments and remaining challenges in modeling infectious diseases with a focus on heterogeneity and aggregation. The goal is to provide attendees with valuable insights into the significance of incorporating heterogeneity into models and effective ways to address associated challenges. 


\vskip\medskipamount

  \leaders\vrule width \textwidth\vskip0.4pt

  \vskip\medskipamount

  \nointerlineskip

  \pagebreak[2]

\end{absolutelynopagebreak}

\begin{absolutelynopagebreak}
\subsection{Nourridine Siewe - TGF-beta inhibition can overcome cancer primary resistance to PD-1 blockade: a mathematical model} 

\begin{tabular}{l}
\toprule
Rochester Institute of Technology\\
USA\\
\bottomrule
\end{tabular}

\begin{tabular}{l}
\toprule
Minisymposium presentation\\
(Mathematical modeling and analysis in cancer immunotherapy)\\
\bottomrule
\end{tabular}
\vskip0.5cm

  Mathematical modelling is an important tool for understanding and controlling the spread of infectious diseases. Heterogeneity, which refers to differences in factors such as demographics, behaviour, susceptibility, infectiousness, and disease severity within a population, plays a critical role in disease transmission and control. Incorporating heterogeneity into models can help researchers better understand disease spread across subpopulations and design more targeted control strategies. However, heterogeneous models can be high-dimensional and complex, leading to theoretical challenges in modelling analysis. Moreover, data collections in the field are often in the forms of aggregation, making modeling implementation challenging. In this talk, we will discuss recent developments and remaining challenges in modeling infectious diseases with a focus on heterogeneity and aggregation. The goal is to provide attendees with valuable insights into the significance of incorporating heterogeneity into models and effective ways to address associated challenges. 


\vskip\medskipamount

  \leaders\vrule width \textwidth\vskip0.4pt

  \vskip\medskipamount

  \nointerlineskip

  \pagebreak[2]

\end{absolutelynopagebreak}

\begin{absolutelynopagebreak}
\subsection{Nourridine Siewe - Increase Hemoglobin Level in Severe Malarial Anemia while Controlling Parasitemia: A Mathematical Model} 

\begin{tabular}{l}
\toprule
Rochester Institute of Technology\\
USA\\
\bottomrule
\end{tabular}

\begin{tabular}{l}
\toprule
Minisymposium presentation\\
(Vector-Borne Disease Dynamics)\\
\bottomrule
\end{tabular}
\vskip0.5cm

  Mathematical modelling is an important tool for understanding and controlling the spread of infectious diseases. Heterogeneity, which refers to differences in factors such as demographics, behaviour, susceptibility, infectiousness, and disease severity within a population, plays a critical role in disease transmission and control. Incorporating heterogeneity into models can help researchers better understand disease spread across subpopulations and design more targeted control strategies. However, heterogeneous models can be high-dimensional and complex, leading to theoretical challenges in modelling analysis. Moreover, data collections in the field are often in the forms of aggregation, making modeling implementation challenging. In this talk, we will discuss recent developments and remaining challenges in modeling infectious diseases with a focus on heterogeneity and aggregation. The goal is to provide attendees with valuable insights into the significance of incorporating heterogeneity into models and effective ways to address associated challenges. 


\vskip\medskipamount

  \leaders\vrule width \textwidth\vskip0.4pt

  \vskip\medskipamount

  \nointerlineskip

  \pagebreak[2]

\end{absolutelynopagebreak}

\begin{absolutelynopagebreak}
\subsection{Nourridine Siewe - Breast cancer exosomal microRNAs facilitate pre-metastatic niche formation in the bone: A mathematical model} 

\begin{tabular}{l}
\toprule
Rochester Institute of Technology\\
USA\\
\bottomrule
\end{tabular}

\begin{tabular}{l}
\toprule
Minisymposium presentation\\
(Within-host and between-host mathematical models of biological dynamics)\\
\bottomrule
\end{tabular}
\vskip0.5cm

  Mathematical modelling is an important tool for understanding and controlling the spread of infectious diseases. Heterogeneity, which refers to differences in factors such as demographics, behaviour, susceptibility, infectiousness, and disease severity within a population, plays a critical role in disease transmission and control. Incorporating heterogeneity into models can help researchers better understand disease spread across subpopulations and design more targeted control strategies. However, heterogeneous models can be high-dimensional and complex, leading to theoretical challenges in modelling analysis. Moreover, data collections in the field are often in the forms of aggregation, making modeling implementation challenging. In this talk, we will discuss recent developments and remaining challenges in modeling infectious diseases with a focus on heterogeneity and aggregation. The goal is to provide attendees with valuable insights into the significance of incorporating heterogeneity into models and effective ways to address associated challenges. 


\vskip\medskipamount

  \leaders\vrule width \textwidth\vskip0.4pt

  \vskip\medskipamount

  \nointerlineskip

  \pagebreak[2]

\end{absolutelynopagebreak}

\begin{absolutelynopagebreak}
\subsection{Stacey Smith? - Coupling the within-host process and between-host transmission of COVID-19 suggests vaccination and school closures are critical} 

\begin{tabular}{l}
\toprule
The University of Ottawa\\
Canada\\
\bottomrule
\end{tabular}

\begin{tabular}{l}
\toprule
Minisymposium presentation\\
(Multiscale models of infectious diseases)\\
\bottomrule
\end{tabular}
\vskip0.5cm

  Mathematical modelling is an important tool for understanding and controlling the spread of infectious diseases. Heterogeneity, which refers to differences in factors such as demographics, behaviour, susceptibility, infectiousness, and disease severity within a population, plays a critical role in disease transmission and control. Incorporating heterogeneity into models can help researchers better understand disease spread across subpopulations and design more targeted control strategies. However, heterogeneous models can be high-dimensional and complex, leading to theoretical challenges in modelling analysis. Moreover, data collections in the field are often in the forms of aggregation, making modeling implementation challenging. In this talk, we will discuss recent developments and remaining challenges in modeling infectious diseases with a focus on heterogeneity and aggregation. The goal is to provide attendees with valuable insights into the significance of incorporating heterogeneity into models and effective ways to address associated challenges. 


\vskip\medskipamount

  \leaders\vrule width \textwidth\vskip0.4pt

  \vskip\medskipamount

  \nointerlineskip

  \pagebreak[2]

\end{absolutelynopagebreak}

\begin{absolutelynopagebreak}
\subsection{Tracy Stepien - Deciphering Glioma Microenvironment Entry Mechanisms of Myeloid-Derived Suppressor Cells} 

\begin{tabular}{l}
\toprule
University of Florida\\
USA\\
\bottomrule
\end{tabular}

\begin{tabular}{l}
\toprule
Minisymposium presentation\\
(Modelling the Cancer Microenvironment)\\
\bottomrule
\end{tabular}
\vskip0.5cm

  Mathematical modelling is an important tool for understanding and controlling the spread of infectious diseases. Heterogeneity, which refers to differences in factors such as demographics, behaviour, susceptibility, infectiousness, and disease severity within a population, plays a critical role in disease transmission and control. Incorporating heterogeneity into models can help researchers better understand disease spread across subpopulations and design more targeted control strategies. However, heterogeneous models can be high-dimensional and complex, leading to theoretical challenges in modelling analysis. Moreover, data collections in the field are often in the forms of aggregation, making modeling implementation challenging. In this talk, we will discuss recent developments and remaining challenges in modeling infectious diseases with a focus on heterogeneity and aggregation. The goal is to provide attendees with valuable insights into the significance of incorporating heterogeneity into models and effective ways to address associated challenges. 


\vskip\medskipamount

  \leaders\vrule width \textwidth\vskip0.4pt

  \vskip\medskipamount

  \nointerlineskip

  \pagebreak[2]

\end{absolutelynopagebreak}

\begin{absolutelynopagebreak}
\subsection{Yasuhiro Takeuchi - Stability analysis of a single-species logistic model with time delay and constant inflow} 

\begin{tabular}{l}
\toprule
Aoyama Gakuin University\\
Japan\\
\bottomrule
\end{tabular}

\begin{tabular}{l}
\toprule
Contributed presentation\\
\bottomrule
\end{tabular}
\vskip0.5cm

  Mathematical modelling is an important tool for understanding and controlling the spread of infectious diseases. Heterogeneity, which refers to differences in factors such as demographics, behaviour, susceptibility, infectiousness, and disease severity within a population, plays a critical role in disease transmission and control. Incorporating heterogeneity into models can help researchers better understand disease spread across subpopulations and design more targeted control strategies. However, heterogeneous models can be high-dimensional and complex, leading to theoretical challenges in modelling analysis. Moreover, data collections in the field are often in the forms of aggregation, making modeling implementation challenging. In this talk, we will discuss recent developments and remaining challenges in modeling infectious diseases with a focus on heterogeneity and aggregation. The goal is to provide attendees with valuable insights into the significance of incorporating heterogeneity into models and effective ways to address associated challenges. 


\vskip\medskipamount

  \leaders\vrule width \textwidth\vskip0.4pt

  \vskip\medskipamount

  \nointerlineskip

  \pagebreak[2]

\end{absolutelynopagebreak}

\begin{absolutelynopagebreak}
\subsection{Ryan Thiessen - Travelling waves of a new glioma invasion model} 

\begin{tabular}{l}
\toprule
University of Alberta\\
Canada\\
\bottomrule
\end{tabular}

\begin{tabular}{l}
\toprule
Minisymposium presentation\\
(Modelling the Cancer Microenvironment)\\
\bottomrule
\end{tabular}
\vskip0.5cm

  Mathematical modelling is an important tool for understanding and controlling the spread of infectious diseases. Heterogeneity, which refers to differences in factors such as demographics, behaviour, susceptibility, infectiousness, and disease severity within a population, plays a critical role in disease transmission and control. Incorporating heterogeneity into models can help researchers better understand disease spread across subpopulations and design more targeted control strategies. However, heterogeneous models can be high-dimensional and complex, leading to theoretical challenges in modelling analysis. Moreover, data collections in the field are often in the forms of aggregation, making modeling implementation challenging. In this talk, we will discuss recent developments and remaining challenges in modeling infectious diseases with a focus on heterogeneity and aggregation. The goal is to provide attendees with valuable insights into the significance of incorporating heterogeneity into models and effective ways to address associated challenges. 


\vskip\medskipamount

  \leaders\vrule width \textwidth\vskip0.4pt

  \vskip\medskipamount

  \nointerlineskip

  \pagebreak[2]

\end{absolutelynopagebreak}

\begin{absolutelynopagebreak}
\subsection{Necibe Tuncer - Determining Reliable Parameter Estimates for Within-host and Within-vector models of Zika Virus} 

\begin{tabular}{l}
\toprule
Florida Atlantic University\\
USA\\
\bottomrule
\end{tabular}

\begin{tabular}{l}
\toprule
Minisymposium presentation\\
(Vector-Borne Disease Dynamics)\\
\bottomrule
\end{tabular}
\vskip0.5cm

  Mathematical modelling is an important tool for understanding and controlling the spread of infectious diseases. Heterogeneity, which refers to differences in factors such as demographics, behaviour, susceptibility, infectiousness, and disease severity within a population, plays a critical role in disease transmission and control. Incorporating heterogeneity into models can help researchers better understand disease spread across subpopulations and design more targeted control strategies. However, heterogeneous models can be high-dimensional and complex, leading to theoretical challenges in modelling analysis. Moreover, data collections in the field are often in the forms of aggregation, making modeling implementation challenging. In this talk, we will discuss recent developments and remaining challenges in modeling infectious diseases with a focus on heterogeneity and aggregation. The goal is to provide attendees with valuable insights into the significance of incorporating heterogeneity into models and effective ways to address associated challenges. 


\vskip\medskipamount

  \leaders\vrule width \textwidth\vskip0.4pt

  \vskip\medskipamount

  \nointerlineskip

  \pagebreak[2]

\end{absolutelynopagebreak}

\begin{absolutelynopagebreak}
\subsection{Necibe Tuncer - Immuno-epidemiological co-a ection model of HIV infection and opioid addiction} 

\begin{tabular}{l}
\toprule
Florida Atlantic University\\
USA\\
\bottomrule
\end{tabular}

\begin{tabular}{l}
\toprule
Minisymposium presentation\\
(Within-host and between-host mathematical models of biological dynamics)\\
\bottomrule
\end{tabular}
\vskip0.5cm

  Mathematical modelling is an important tool for understanding and controlling the spread of infectious diseases. Heterogeneity, which refers to differences in factors such as demographics, behaviour, susceptibility, infectiousness, and disease severity within a population, plays a critical role in disease transmission and control. Incorporating heterogeneity into models can help researchers better understand disease spread across subpopulations and design more targeted control strategies. However, heterogeneous models can be high-dimensional and complex, leading to theoretical challenges in modelling analysis. Moreover, data collections in the field are often in the forms of aggregation, making modeling implementation challenging. In this talk, we will discuss recent developments and remaining challenges in modeling infectious diseases with a focus on heterogeneity and aggregation. The goal is to provide attendees with valuable insights into the significance of incorporating heterogeneity into models and effective ways to address associated challenges. 


\vskip\medskipamount

  \leaders\vrule width \textwidth\vskip0.4pt

  \vskip\medskipamount

  \nointerlineskip

  \pagebreak[2]

\end{absolutelynopagebreak}

\begin{absolutelynopagebreak}
\subsection{Sonja Türpitz - Considering Subpopulations in Modelling Facultative Mutualism Reveals a New Approach to Model Interspecific Interactions} 

\begin{tabular}{l}
\toprule
Friedrich Schiller University Jena, Germany\\
Germany\\
\bottomrule
\end{tabular}

\begin{tabular}{l}
\toprule
Contributed presentation\\
\bottomrule
\end{tabular}
\vskip0.5cm

  Mathematical modelling is an important tool for understanding and controlling the spread of infectious diseases. Heterogeneity, which refers to differences in factors such as demographics, behaviour, susceptibility, infectiousness, and disease severity within a population, plays a critical role in disease transmission and control. Incorporating heterogeneity into models can help researchers better understand disease spread across subpopulations and design more targeted control strategies. However, heterogeneous models can be high-dimensional and complex, leading to theoretical challenges in modelling analysis. Moreover, data collections in the field are often in the forms of aggregation, making modeling implementation challenging. In this talk, we will discuss recent developments and remaining challenges in modeling infectious diseases with a focus on heterogeneity and aggregation. The goal is to provide attendees with valuable insights into the significance of incorporating heterogeneity into models and effective ways to address associated challenges. 


\vskip\medskipamount

  \leaders\vrule width \textwidth\vskip0.4pt

  \vskip\medskipamount

  \nointerlineskip

  \pagebreak[2]

\end{absolutelynopagebreak}

\begin{absolutelynopagebreak}
\subsection{Marie Betsy Varughese - Incorporating Health Seeking Behaviour in a Deterministic Model for Influenza} 

\begin{tabular}{l}
\toprule
University of Alberta\\
USA\\
\bottomrule
\end{tabular}

\begin{tabular}{l}
\toprule
Minisymposium presentation\\
(Recent Advances in Modelling Infectious Diseases)\\
\bottomrule
\end{tabular}
\vskip0.5cm

  Mathematical modelling is an important tool for understanding and controlling the spread of infectious diseases. Heterogeneity, which refers to differences in factors such as demographics, behaviour, susceptibility, infectiousness, and disease severity within a population, plays a critical role in disease transmission and control. Incorporating heterogeneity into models can help researchers better understand disease spread across subpopulations and design more targeted control strategies. However, heterogeneous models can be high-dimensional and complex, leading to theoretical challenges in modelling analysis. Moreover, data collections in the field are often in the forms of aggregation, making modeling implementation challenging. In this talk, we will discuss recent developments and remaining challenges in modeling infectious diseases with a focus on heterogeneity and aggregation. The goal is to provide attendees with valuable insights into the significance of incorporating heterogeneity into models and effective ways to address associated challenges. 


\vskip\medskipamount

  \leaders\vrule width \textwidth\vskip0.4pt

  \vskip\medskipamount

  \nointerlineskip

  \pagebreak[2]

\end{absolutelynopagebreak}

\begin{absolutelynopagebreak}
\subsection{Jorge Velasco-Hernandez - Modeling a traffic light warning system for acute respiratory infections} 

\begin{tabular}{l}
\toprule
Universidad nacional Autónoma de México\\
Mexico\\
\bottomrule
\end{tabular}

\begin{tabular}{l}
\toprule
Minisymposium presentation\\
(Recent Advances in Modelling Infectious Diseases)\\
\bottomrule
\end{tabular}
\vskip0.5cm

  Mathematical modelling is an important tool for understanding and controlling the spread of infectious diseases. Heterogeneity, which refers to differences in factors such as demographics, behaviour, susceptibility, infectiousness, and disease severity within a population, plays a critical role in disease transmission and control. Incorporating heterogeneity into models can help researchers better understand disease spread across subpopulations and design more targeted control strategies. However, heterogeneous models can be high-dimensional and complex, leading to theoretical challenges in modelling analysis. Moreover, data collections in the field are often in the forms of aggregation, making modeling implementation challenging. In this talk, we will discuss recent developments and remaining challenges in modeling infectious diseases with a focus on heterogeneity and aggregation. The goal is to provide attendees with valuable insights into the significance of incorporating heterogeneity into models and effective ways to address associated challenges. 


\vskip\medskipamount

  \leaders\vrule width \textwidth\vskip0.4pt

  \vskip\medskipamount

  \nointerlineskip

  \pagebreak[2]

\end{absolutelynopagebreak}

\begin{absolutelynopagebreak}
\subsection{Jorge Velasco-Hernandez - The Ross-Mcdonald model revisited: linking transmission and within-host dynamics} 

\begin{tabular}{l}
\toprule
Universidad nacional Autónoma de México\\
Mexico\\
\bottomrule
\end{tabular}

\begin{tabular}{l}
\toprule
Minisymposium presentation\\
(Vector-Borne Disease Dynamics)\\
\bottomrule
\end{tabular}
\vskip0.5cm

  Mathematical modelling is an important tool for understanding and controlling the spread of infectious diseases. Heterogeneity, which refers to differences in factors such as demographics, behaviour, susceptibility, infectiousness, and disease severity within a population, plays a critical role in disease transmission and control. Incorporating heterogeneity into models can help researchers better understand disease spread across subpopulations and design more targeted control strategies. However, heterogeneous models can be high-dimensional and complex, leading to theoretical challenges in modelling analysis. Moreover, data collections in the field are often in the forms of aggregation, making modeling implementation challenging. In this talk, we will discuss recent developments and remaining challenges in modeling infectious diseases with a focus on heterogeneity and aggregation. The goal is to provide attendees with valuable insights into the significance of incorporating heterogeneity into models and effective ways to address associated challenges. 


\vskip\medskipamount

  \leaders\vrule width \textwidth\vskip0.4pt

  \vskip\medskipamount

  \nointerlineskip

  \pagebreak[2]

\end{absolutelynopagebreak}

\begin{absolutelynopagebreak}
\subsection{Amy Veprauskas - The interplay between dispersal and Allee effects in discrete-time population models} 

\begin{tabular}{l}
\toprule
University of Louisiana at Lafayette\\
USA\\
\bottomrule
\end{tabular}

\begin{tabular}{l}
\toprule
Minisymposium presentation\\
(Ecological and Epidemiological Models with Dispersal)\\
\bottomrule
\end{tabular}
\vskip0.5cm

  Mathematical modelling is an important tool for understanding and controlling the spread of infectious diseases. Heterogeneity, which refers to differences in factors such as demographics, behaviour, susceptibility, infectiousness, and disease severity within a population, plays a critical role in disease transmission and control. Incorporating heterogeneity into models can help researchers better understand disease spread across subpopulations and design more targeted control strategies. However, heterogeneous models can be high-dimensional and complex, leading to theoretical challenges in modelling analysis. Moreover, data collections in the field are often in the forms of aggregation, making modeling implementation challenging. In this talk, we will discuss recent developments and remaining challenges in modeling infectious diseases with a focus on heterogeneity and aggregation. The goal is to provide attendees with valuable insights into the significance of incorporating heterogeneity into models and effective ways to address associated challenges. 


\vskip\medskipamount

  \leaders\vrule width \textwidth\vskip0.4pt

  \vskip\medskipamount

  \nointerlineskip

  \pagebreak[2]

\end{absolutelynopagebreak}

\begin{absolutelynopagebreak}
\subsection{Amy Veprauskas - Pathogen dynamic in a tick-host system: A discrete-time modeling approach} 

\begin{tabular}{l}
\toprule
University of Louisiana at Lafayette\\
USA\\
\bottomrule
\end{tabular}

\begin{tabular}{l}
\toprule
Minisymposium presentation\\
(Recent Advances in Modelling Infectious Diseases)\\
\bottomrule
\end{tabular}
\vskip0.5cm

  Mathematical modelling is an important tool for understanding and controlling the spread of infectious diseases. Heterogeneity, which refers to differences in factors such as demographics, behaviour, susceptibility, infectiousness, and disease severity within a population, plays a critical role in disease transmission and control. Incorporating heterogeneity into models can help researchers better understand disease spread across subpopulations and design more targeted control strategies. However, heterogeneous models can be high-dimensional and complex, leading to theoretical challenges in modelling analysis. Moreover, data collections in the field are often in the forms of aggregation, making modeling implementation challenging. In this talk, we will discuss recent developments and remaining challenges in modeling infectious diseases with a focus on heterogeneity and aggregation. The goal is to provide attendees with valuable insights into the significance of incorporating heterogeneity into models and effective ways to address associated challenges. 


\vskip\medskipamount

  \leaders\vrule width \textwidth\vskip0.4pt

  \vskip\medskipamount

  \nointerlineskip

  \pagebreak[2]

\end{absolutelynopagebreak}

\begin{absolutelynopagebreak}
\subsection{Ren-Yi Wang - Analysis of A Countable-Type Branching Process Model for the Tug-of-War Cancer Cell Dynamics} 

\begin{tabular}{l}
\toprule
Rice University\\
USA\\
\bottomrule
\end{tabular}

\begin{tabular}{l}
\toprule
Minisymposium presentation\\
(Stochastic population models: Theory and applications in Cancer Research)\\
\bottomrule
\end{tabular}
\vskip0.5cm

  Mathematical modelling is an important tool for understanding and controlling the spread of infectious diseases. Heterogeneity, which refers to differences in factors such as demographics, behaviour, susceptibility, infectiousness, and disease severity within a population, plays a critical role in disease transmission and control. Incorporating heterogeneity into models can help researchers better understand disease spread across subpopulations and design more targeted control strategies. However, heterogeneous models can be high-dimensional and complex, leading to theoretical challenges in modelling analysis. Moreover, data collections in the field are often in the forms of aggregation, making modeling implementation challenging. In this talk, we will discuss recent developments and remaining challenges in modeling infectious diseases with a focus on heterogeneity and aggregation. The goal is to provide attendees with valuable insights into the significance of incorporating heterogeneity into models and effective ways to address associated challenges. 


\vskip\medskipamount

  \leaders\vrule width \textwidth\vskip0.4pt

  \vskip\medskipamount

  \nointerlineskip

  \pagebreak[2]

\end{absolutelynopagebreak}

\begin{absolutelynopagebreak}
\subsection{Xuyuan Wang - Detecting and Resolving Nonidentifiability In Infectious Diseases Modeling} 

\begin{tabular}{l}
\toprule
University of Alberta\\
Canada\\
\bottomrule
\end{tabular}

\begin{tabular}{l}
\toprule
Minisymposium presentation\\
(Recent Advances in Modelling Infectious Diseases)\\
\bottomrule
\end{tabular}
\vskip0.5cm

  Mathematical modelling is an important tool for understanding and controlling the spread of infectious diseases. Heterogeneity, which refers to differences in factors such as demographics, behaviour, susceptibility, infectiousness, and disease severity within a population, plays a critical role in disease transmission and control. Incorporating heterogeneity into models can help researchers better understand disease spread across subpopulations and design more targeted control strategies. However, heterogeneous models can be high-dimensional and complex, leading to theoretical challenges in modelling analysis. Moreover, data collections in the field are often in the forms of aggregation, making modeling implementation challenging. In this talk, we will discuss recent developments and remaining challenges in modeling infectious diseases with a focus on heterogeneity and aggregation. The goal is to provide attendees with valuable insights into the significance of incorporating heterogeneity into models and effective ways to address associated challenges. 


\vskip\medskipamount

  \leaders\vrule width \textwidth\vskip0.4pt

  \vskip\medskipamount

  \nointerlineskip

  \pagebreak[2]

\end{absolutelynopagebreak}

\begin{absolutelynopagebreak}
\subsection{Kathleen Wilkie - Modelling the Evolution of the Immune Response to Cancer} 

\begin{tabular}{l}
\toprule
Toronto Metropolitan University\\
Canada\\
\bottomrule
\end{tabular}

\begin{tabular}{l}
\toprule
Minisymposium presentation\\
(Mathematical and computational approaches to modelling immunology)\\
\bottomrule
\end{tabular}
\vskip0.5cm

  Mathematical modelling is an important tool for understanding and controlling the spread of infectious diseases. Heterogeneity, which refers to differences in factors such as demographics, behaviour, susceptibility, infectiousness, and disease severity within a population, plays a critical role in disease transmission and control. Incorporating heterogeneity into models can help researchers better understand disease spread across subpopulations and design more targeted control strategies. However, heterogeneous models can be high-dimensional and complex, leading to theoretical challenges in modelling analysis. Moreover, data collections in the field are often in the forms of aggregation, making modeling implementation challenging. In this talk, we will discuss recent developments and remaining challenges in modeling infectious diseases with a focus on heterogeneity and aggregation. The goal is to provide attendees with valuable insights into the significance of incorporating heterogeneity into models and effective ways to address associated challenges. 


\vskip\medskipamount

  \leaders\vrule width \textwidth\vskip0.4pt

  \vskip\medskipamount

  \nointerlineskip

  \pagebreak[2]

\end{absolutelynopagebreak}

\begin{absolutelynopagebreak}
\subsection{Kathleen Wilkie - Modelling Radiation Cancer Treatment with Ordinary and Fractional Differential Equations} 

\begin{tabular}{l}
\toprule
Toronto Metropolitan University\\
Canada\\
\bottomrule
\end{tabular}

\begin{tabular}{l}
\toprule
Minisymposium presentation\\
(Modelling the Cancer Microenvironment)\\
\bottomrule
\end{tabular}
\vskip0.5cm

  Mathematical modelling is an important tool for understanding and controlling the spread of infectious diseases. Heterogeneity, which refers to differences in factors such as demographics, behaviour, susceptibility, infectiousness, and disease severity within a population, plays a critical role in disease transmission and control. Incorporating heterogeneity into models can help researchers better understand disease spread across subpopulations and design more targeted control strategies. However, heterogeneous models can be high-dimensional and complex, leading to theoretical challenges in modelling analysis. Moreover, data collections in the field are often in the forms of aggregation, making modeling implementation challenging. In this talk, we will discuss recent developments and remaining challenges in modeling infectious diseases with a focus on heterogeneity and aggregation. The goal is to provide attendees with valuable insights into the significance of incorporating heterogeneity into models and effective ways to address associated challenges. 


\vskip\medskipamount

  \leaders\vrule width \textwidth\vskip0.4pt

  \vskip\medskipamount

  \nointerlineskip

  \pagebreak[2]

\end{absolutelynopagebreak}

\begin{absolutelynopagebreak}
\subsection{Pei Yuan - Modelling for informing public health policy on prevention and control of COVID-19 epidemics in Toronto, Canada} 

\begin{tabular}{l}
\toprule
York University\\
Canada\\
\bottomrule
\end{tabular}

\begin{tabular}{l}
\toprule
Minisymposium presentation\\
(Recent Advances in Modelling Infectious Diseases)\\
\bottomrule
\end{tabular}
\vskip0.5cm

  Mathematical modelling is an important tool for understanding and controlling the spread of infectious diseases. Heterogeneity, which refers to differences in factors such as demographics, behaviour, susceptibility, infectiousness, and disease severity within a population, plays a critical role in disease transmission and control. Incorporating heterogeneity into models can help researchers better understand disease spread across subpopulations and design more targeted control strategies. However, heterogeneous models can be high-dimensional and complex, leading to theoretical challenges in modelling analysis. Moreover, data collections in the field are often in the forms of aggregation, making modeling implementation challenging. In this talk, we will discuss recent developments and remaining challenges in modeling infectious diseases with a focus on heterogeneity and aggregation. The goal is to provide attendees with valuable insights into the significance of incorporating heterogeneity into models and effective ways to address associated challenges. 


\vskip\medskipamount

  \leaders\vrule width \textwidth\vskip0.4pt

  \vskip\medskipamount

  \nointerlineskip

  \pagebreak[2]

\end{absolutelynopagebreak}

\begin{absolutelynopagebreak}
\subsection{Veronika Zarnitsyna - Competing Heterogeneities in Vaccine Effectiveness Estimation} 

\begin{tabular}{l}
\toprule
Department of Microbiology and Immunology, Emory University School of Medicine\\
USA\\
\bottomrule
\end{tabular}

\begin{tabular}{l}
\toprule
Minisymposium presentation\\
(Bridging the scale from within-host to epidemic models)\\
\bottomrule
\end{tabular}
\vskip0.5cm

  Mathematical modelling is an important tool for understanding and controlling the spread of infectious diseases. Heterogeneity, which refers to differences in factors such as demographics, behaviour, susceptibility, infectiousness, and disease severity within a population, plays a critical role in disease transmission and control. Incorporating heterogeneity into models can help researchers better understand disease spread across subpopulations and design more targeted control strategies. However, heterogeneous models can be high-dimensional and complex, leading to theoretical challenges in modelling analysis. Moreover, data collections in the field are often in the forms of aggregation, making modeling implementation challenging. In this talk, we will discuss recent developments and remaining challenges in modeling infectious diseases with a focus on heterogeneity and aggregation. The goal is to provide attendees with valuable insights into the significance of incorporating heterogeneity into models and effective ways to address associated challenges. 


\vskip\medskipamount

  \leaders\vrule width \textwidth\vskip0.4pt

  \vskip\medskipamount

  \nointerlineskip

  \pagebreak[2]

\end{absolutelynopagebreak}

\begin{absolutelynopagebreak}
\subsection{Huaiping Zhu - A two-stage model with distributed delay for mosquito population dynamics} 

\begin{tabular}{l}
\toprule
York University\\
Canada\\
\bottomrule
\end{tabular}

\begin{tabular}{l}
\toprule
Minisymposium presentation\\
(Delay-differential equations in applications)\\
\bottomrule
\end{tabular}
\vskip0.5cm

  Mathematical modelling is an important tool for understanding and controlling the spread of infectious diseases. Heterogeneity, which refers to differences in factors such as demographics, behaviour, susceptibility, infectiousness, and disease severity within a population, plays a critical role in disease transmission and control. Incorporating heterogeneity into models can help researchers better understand disease spread across subpopulations and design more targeted control strategies. However, heterogeneous models can be high-dimensional and complex, leading to theoretical challenges in modelling analysis. Moreover, data collections in the field are often in the forms of aggregation, making modeling implementation challenging. In this talk, we will discuss recent developments and remaining challenges in modeling infectious diseases with a focus on heterogeneity and aggregation. The goal is to provide attendees with valuable insights into the significance of incorporating heterogeneity into models and effective ways to address associated challenges. 


\vskip\medskipamount

  \leaders\vrule width \textwidth\vskip0.4pt

  \vskip\medskipamount

  \nointerlineskip

  \pagebreak[2]

\end{absolutelynopagebreak}

\begin{absolutelynopagebreak}
\subsection{Huaiping Zhu - Predictive modelling and forecasting of the mosquito abundance and risk of West Nile virus in Ontario Canada} 

\begin{tabular}{l}
\toprule
York University\\
Canada\\
\bottomrule
\end{tabular}

\begin{tabular}{l}
\toprule
Minisymposium presentation\\
(Vector-Borne Disease Dynamics)\\
\bottomrule
\end{tabular}
\vskip0.5cm

  Mathematical modelling is an important tool for understanding and controlling the spread of infectious diseases. Heterogeneity, which refers to differences in factors such as demographics, behaviour, susceptibility, infectiousness, and disease severity within a population, plays a critical role in disease transmission and control. Incorporating heterogeneity into models can help researchers better understand disease spread across subpopulations and design more targeted control strategies. However, heterogeneous models can be high-dimensional and complex, leading to theoretical challenges in modelling analysis. Moreover, data collections in the field are often in the forms of aggregation, making modeling implementation challenging. In this talk, we will discuss recent developments and remaining challenges in modeling infectious diseases with a focus on heterogeneity and aggregation. The goal is to provide attendees with valuable insights into the significance of incorporating heterogeneity into models and effective ways to address associated challenges. 


\vskip\medskipamount

  \leaders\vrule width \textwidth\vskip0.4pt

  \vskip\medskipamount

  \nointerlineskip

  \pagebreak[2]

\end{absolutelynopagebreak}

\begin{absolutelynopagebreak}
\subsection{Pauline van den Driessche - Disease-Induced Hydra Effect} 

\begin{tabular}{l}
\toprule
University of Victoria, BC\\
Canada\\
\bottomrule
\end{tabular}

\begin{tabular}{l}
\toprule
Minisymposium presentation\\
(Recent Advances in Modelling Infectious Diseases)\\
\bottomrule
\end{tabular}
\vskip0.5cm

  Mathematical modelling is an important tool for understanding and controlling the spread of infectious diseases. Heterogeneity, which refers to differences in factors such as demographics, behaviour, susceptibility, infectiousness, and disease severity within a population, plays a critical role in disease transmission and control. Incorporating heterogeneity into models can help researchers better understand disease spread across subpopulations and design more targeted control strategies. However, heterogeneous models can be high-dimensional and complex, leading to theoretical challenges in modelling analysis. Moreover, data collections in the field are often in the forms of aggregation, making modeling implementation challenging. In this talk, we will discuss recent developments and remaining challenges in modeling infectious diseases with a focus on heterogeneity and aggregation. The goal is to provide attendees with valuable insights into the significance of incorporating heterogeneity into models and effective ways to address associated challenges. 


\vskip\medskipamount

  \leaders\vrule width \textwidth\vskip0.4pt

  \vskip\medskipamount

  \nointerlineskip

  \pagebreak[2]

\end{absolutelynopagebreak}

\newpage

\hypertarget{list-of-participants}{%
\section{List of participants}\label{list-of-participants}}

\begin{absolutelynopagebreak}Azmy   Ackleh \newline
\mbox{}\quad  Mathematics \newline
\mbox{}\quad  University of Louisiana at Lafayette \newline
\mbox{}\quad  United States \newline
\mbox{}\quad \href{mailto: azmy.ackleh@louisiana.edu }{ azmy.ackleh@louisiana.edu }
\end{absolutelynopagebreak}\vskip0.2cm
\begin{absolutelynopagebreak}Folashade   Agusto \newline
\mbox{}\quad  Ecology and Evolutionary Biology \newline
\mbox{}\quad  University of Kansas \newline
\mbox{}\quad  United States \newline
\mbox{}\quad \href{mailto: fbagusto@ku.edu }{ fbagusto@ku.edu }
\end{absolutelynopagebreak}\vskip0.2cm
\begin{absolutelynopagebreak}Ephraim   Agyingi \newline
\mbox{}\quad  Mathematical Sciences \newline
\mbox{}\quad  Rochester Institute of Technology \newline
\mbox{}\quad  United States \newline
\mbox{}\quad \href{mailto: eoasma@rit.edu }{ eoasma@rit.edu }
\end{absolutelynopagebreak}\vskip0.2cm
\begin{absolutelynopagebreak}Vitalii   Akimenko \newline
\mbox{}\quad  Mathematics \newline
\mbox{}\quad  University of Manitoba \newline
\mbox{}\quad  Canada \newline
\mbox{}\quad \href{mailto: vitaliiakm@gmail.com }{ vitaliiakm@gmail.com }
\end{absolutelynopagebreak}\vskip0.2cm
\begin{absolutelynopagebreak}Asami   Anzai \newline
\mbox{}\quad  Graduate School of Medicine \newline
\mbox{}\quad  Kyoto University \newline
\mbox{}\quad  Japan \newline
\mbox{}\quad \href{mailto: anzai.asami.43c@st.kyoto-u.ac.jp }{ anzai.asami.43c@st.kyoto-u.ac.jp }
\end{absolutelynopagebreak}\vskip0.2cm
\begin{absolutelynopagebreak}Julien   Arino \newline
\mbox{}\quad  Mathematics \newline
\mbox{}\quad  University of Manitoba \newline
\mbox{}\quad  Canada \newline
\mbox{}\quad \href{mailto: julien.arino@umanitoba.ca }{ julien.arino@umanitoba.ca }
\end{absolutelynopagebreak}\vskip0.2cm
\begin{absolutelynopagebreak}Joseph   Baafi \newline
\mbox{}\quad  Biology \newline
\mbox{}\quad  Memorial University \newline
\mbox{}\quad  Canada \newline
\mbox{}\quad \href{mailto: jbaafi@mun.ca }{ jbaafi@mun.ca }
\end{absolutelynopagebreak}\vskip0.2cm
\begin{absolutelynopagebreak}Jacques   Bélair \newline
\mbox{}\quad  Mathematics and Statistics \newline
\mbox{}\quad  Université de Montréal \newline
\mbox{}\quad  Canada \newline
\mbox{}\quad \href{mailto: jacques.belair@umontreal.ca }{ jacques.belair@umontreal.ca }
\end{absolutelynopagebreak}\vskip0.2cm
\begin{absolutelynopagebreak}Ranjini   Bhattacharya \newline
\mbox{}\quad  Integrated Mathematical Oncology \newline
\mbox{}\quad  Moffitt Cancer Center \newline
\mbox{}\quad  United States \newline
\mbox{}\quad \href{mailto: ranjini.bhattacharya@moffitt.org }{ ranjini.bhattacharya@moffitt.org }
\end{absolutelynopagebreak}\vskip0.2cm
\begin{absolutelynopagebreak}Tijotop Ahmed    Binjibon \newline
\mbox{}\quad  Mathematics \newline
\mbox{}\quad  University of Manitoba \newline
\mbox{}\quad  Canada \newline
\mbox{}\quad \href{mailto: binjibot@myumanitoba.ca }{ binjibot@myumanitoba.ca }
\end{absolutelynopagebreak}\vskip0.2cm
\begin{absolutelynopagebreak}Amanda   Bleichrodt \newline
\mbox{}\quad  Population Health Sciences \newline
\mbox{}\quad  Georgia State University \newline
\mbox{}\quad  United States \newline
\mbox{}\quad \href{mailto: ableichrodt1@student.gsu.edu }{ ableichrodt1@student.gsu.edu }
\end{absolutelynopagebreak}\vskip0.2cm
\begin{absolutelynopagebreak}Ernesto Augusto   Bueno da Fonseca Lima \newline
\mbox{}\quad  Oden Institute \newline
\mbox{}\quad  The University of Texas at Austin \newline
\mbox{}\quad  United States \newline
\mbox{}\quad \href{mailto: ernesto.lima@utexas.edu }{ ernesto.lima@utexas.edu }
\end{absolutelynopagebreak}\vskip0.2cm
\begin{absolutelynopagebreak}Anuraag   Bukkuri \newline
\mbox{}\quad  Integrated Mathematical Oncology \newline
\mbox{}\quad  Moffitt Cancer Center \newline
\mbox{}\quad  United States \newline
\mbox{}\quad \href{mailto: anuraag.bukkuri@moffitt.org }{ anuraag.bukkuri@moffitt.org }
\end{absolutelynopagebreak}\vskip0.2cm
\begin{absolutelynopagebreak}Robert Stephen   Cantrell \newline
\mbox{}\quad  Mathematics \newline
\mbox{}\quad  University of Miami \newline
\mbox{}\quad  United States \newline
\mbox{}\quad \href{mailto: rsc@math.miami.edu }{ rsc@math.miami.edu }
\end{absolutelynopagebreak}\vskip0.2cm
\begin{absolutelynopagebreak}Erwing   Cardozo-Ojeda \newline
\mbox{}\quad  Vaccine and Infectious Disease Division \newline
\mbox{}\quad  Fred Hutchinson Cancer Center \newline
\mbox{}\quad  United States \newline
\mbox{}\quad \href{mailto: ecojeda@fredhutch.org }{ ecojeda@fredhutch.org }
\end{absolutelynopagebreak}\vskip0.2cm
\begin{absolutelynopagebreak}Bernard   Cazelles \newline
\mbox{}\quad  UMMISCO \newline
\mbox{}\quad  IRD Sorbonne Université \newline
\mbox{}\quad  France \newline
\mbox{}\quad \href{mailto: cazelles@biologie.ens.fr }{ cazelles@biologie.ens.fr }
\end{absolutelynopagebreak}\vskip0.2cm
\begin{absolutelynopagebreak}Stanca   Ciupe \newline
\mbox{}\quad  Mathematics \newline
\mbox{}\quad  Virginia Tech \newline
\mbox{}\quad  United States \newline
\mbox{}\quad \href{mailto: stanca@vt.edu }{ stanca@vt.edu }
\end{absolutelynopagebreak}\vskip0.2cm
\begin{absolutelynopagebreak}Adriana-Stefania   Ciupeanu \newline
\mbox{}\quad  Mathematics \newline
\mbox{}\quad  University of Manitoba \newline
\mbox{}\quad  Canada \newline
\mbox{}\quad \href{mailto: ciupeana@myumanitoba.ca }{ ciupeana@myumanitoba.ca }
\end{absolutelynopagebreak}\vskip0.2cm
\begin{absolutelynopagebreak}Jessica   Conway \newline
\mbox{}\quad  Mathematics \newline
\mbox{}\quad  Penn State \newline
\mbox{}\quad  United States \newline
\mbox{}\quad \href{mailto: jmc90@psu.edu }{ jmc90@psu.edu }
\end{absolutelynopagebreak}\vskip0.2cm
\begin{absolutelynopagebreak}Morgan   Craig \newline
\mbox{}\quad  Immune disorders and cancer \& Mathematics and Statistics \newline
\mbox{}\quad  Sainte-Justine University Hospital Research Centre \& Université de Montréal \newline
\mbox{}\quad  Canada \newline
\mbox{}\quad \href{mailto: morgan.craig@umontreal.ca }{ morgan.craig@umontreal.ca }
\end{absolutelynopagebreak}\vskip0.2cm
\begin{absolutelynopagebreak}Jim   Cushing \newline
\mbox{}\quad  Mathematics \newline
\mbox{}\quad  University of Arizona \newline
\mbox{}\quad  United States \newline
\mbox{}\quad \href{mailto: cushing@math.arizona.edu }{ cushing@math.arizona.edu }
\end{absolutelynopagebreak}\vskip0.2cm
\begin{absolutelynopagebreak}Tanuja   Das \newline
\mbox{}\quad  Mathematics and Statistics \newline
\mbox{}\quad  University of New Brunswick \newline
\mbox{}\quad  Canada \newline
\mbox{}\quad \href{mailto: tanujamanidas@gmail.com }{ tanujamanidas@gmail.com }
\end{absolutelynopagebreak}\vskip0.2cm
\begin{absolutelynopagebreak}Xiaoyan   Deng \newline
\mbox{}\quad  Mathematics and Statistics \newline
\mbox{}\quad  Université de Montréal \newline
\mbox{}\quad  Canada \newline
\mbox{}\quad \href{mailto: xiaoyan.deng@umontreal.ca }{ xiaoyan.deng@umontreal.ca }
\end{absolutelynopagebreak}\vskip0.2cm
\begin{absolutelynopagebreak}Clotilde   Djuikem \newline
\mbox{}\quad  BIOCORE \newline
\mbox{}\quad  INRIA Sophia Antipolis \newline
\mbox{}\quad  France \newline
\mbox{}\quad \href{mailto: clotilde.djuikem@inria.fr }{ clotilde.djuikem@inria.fr }
\end{absolutelynopagebreak}\vskip0.2cm
\begin{absolutelynopagebreak}Marisa   Eisenberg \newline
\mbox{}\quad  Epidemiology, Complex Systems \& Mathematics \newline
\mbox{}\quad  University of Michigan, Ann Arbor \newline
\mbox{}\quad  United States \newline
\mbox{}\quad \href{mailto: marisae@umich.edu }{ marisae@umich.edu }
\end{absolutelynopagebreak}\vskip0.2cm
\begin{absolutelynopagebreak}Blessing   Emerenini \newline
\mbox{}\quad  School of Mathematical Sciences \newline
\mbox{}\quad  Rochester Institute of Technology \newline
\mbox{}\quad  United States of America \newline
\mbox{}\quad \href{mailto: boesma@rit.edu }{ boesma@rit.edu }
\end{absolutelynopagebreak}\vskip0.2cm
\begin{absolutelynopagebreak}Guihong   Fan \newline
\mbox{}\quad  Mathematics \newline
\mbox{}\quad  Columbus State University \newline
\mbox{}\quad  United States \newline
\mbox{}\quad \href{mailto: fan\_guihong@columbusstate.edu }{ fan\_guihong@columbusstate.edu }
\end{absolutelynopagebreak}\vskip0.2cm
\begin{absolutelynopagebreak}Suzan   Farhang-Sardroodi \newline
\mbox{}\quad  Mathematics \newline
\mbox{}\quad  University of Manitoba \newline
\mbox{}\quad  Canada \newline
\mbox{}\quad \href{mailto: suzan.farhangsardroodi@umanitoba.ca }{ suzan.farhangsardroodi@umanitoba.ca }
\end{absolutelynopagebreak}\vskip0.2cm
\begin{absolutelynopagebreak}Ghazale   Farjam \newline
\mbox{}\quad  Department of Mathematics \newline
\mbox{}\quad  University of Manitoba \newline
\mbox{}\quad  Canada \newline
\mbox{}\quad \href{mailto: farjamg@myumanitoba.ca }{ farjamg@myumanitoba.ca }
\end{absolutelynopagebreak}\vskip0.2cm
\begin{absolutelynopagebreak}Jonathan   Forde \newline
\mbox{}\quad  Mathematics and Computer Science \newline
\mbox{}\quad  Hobart and William Smith Colleges \newline
\mbox{}\quad  United States \newline
\mbox{}\quad \href{mailto: forde@hws.edu }{ forde@hws.edu }
\end{absolutelynopagebreak}\vskip0.2cm
\begin{absolutelynopagebreak}Samaneh   Gholami \newline
\mbox{}\quad  Mathematics and Statistics \newline
\mbox{}\quad  York University \newline
\mbox{}\quad  Canada \newline
\mbox{}\quad \href{mailto: sama20@yorku.ca }{ sama20@yorku.ca }
\end{absolutelynopagebreak}\vskip0.2cm
\begin{absolutelynopagebreak}Abba   Gumel \newline
\mbox{}\quad  Mathematics \newline
\mbox{}\quad  University of Maryland \newline
\mbox{}\quad  United States \newline
\mbox{}\quad \href{mailto: agumel@umd.edu }{ agumel@umd.edu }
\end{absolutelynopagebreak}\vskip0.2cm
\begin{absolutelynopagebreak}Donglin   Han \newline
\mbox{}\quad  Mathematical and Statistical Sciences \newline
\mbox{}\quad  University of Alberta \newline
\mbox{}\quad  Canada \newline
\mbox{}\quad \href{mailto: donglin3@ualberta.ca }{ donglin3@ualberta.ca }
\end{absolutelynopagebreak}\vskip0.2cm
\begin{absolutelynopagebreak}Md. Mehadi   Hasan \newline
\mbox{}\quad  Mathematics \newline
\mbox{}\quad  University of Manitoba \newline
\mbox{}\quad  Canada \newline
\mbox{}\quad \href{mailto: hasanmm4@myumanitoba.ca }{ hasanmm4@myumanitoba.ca }
\end{absolutelynopagebreak}\vskip0.2cm
\begin{absolutelynopagebreak}Katsuma   Hayashi \newline
\mbox{}\quad  Hygiene \newline
\mbox{}\quad  Kyoto University \newline
\mbox{}\quad  Japan \newline
\mbox{}\quad \href{mailto: hayashi.katsuma.7w@kyoto-u.ac.jp }{ hayashi.katsuma.7w@kyoto-u.ac.jp }
\end{absolutelynopagebreak}\vskip0.2cm
\begin{absolutelynopagebreak}Jane   Heffernan \newline
\mbox{}\quad  Mathematics and Statistics \newline
\mbox{}\quad  York University \newline
\mbox{}\quad  Canada \newline
\mbox{}\quad \href{mailto: jmheffer@yorku.ca }{ jmheffer@yorku.ca }
\end{absolutelynopagebreak}\vskip0.2cm
\begin{absolutelynopagebreak}Esteban A.   Hernandez-Vargas \newline
\mbox{}\quad  Mathematics and Statistical Science \newline
\mbox{}\quad  University of Idaho \newline
\mbox{}\quad  United States \newline
\mbox{}\quad \href{mailto: esteban@uidaho.edu }{ esteban@uidaho.edu }
\end{absolutelynopagebreak}\vskip0.2cm
\begin{absolutelynopagebreak}Thomas   Hillen \newline
\mbox{}\quad  Mathematical and Statistical Sciences \newline
\mbox{}\quad  University of Alberta \newline
\mbox{}\quad  Canada \newline
\mbox{}\quad \href{mailto: thillen@ualberta.ca }{ thillen@ualberta.ca }
\end{absolutelynopagebreak}\vskip0.2cm
\begin{absolutelynopagebreak}Jannatun Irana   Ira \newline
\mbox{}\quad  Mathematics \newline
\mbox{}\quad  University of Manitoba \newline
\mbox{}\quad  Canada \newline
\mbox{}\quad \href{mailto: iraji@myumanitoba.ca }{ iraji@myumanitoba.ca }
\end{absolutelynopagebreak}\vskip0.2cm
\begin{absolutelynopagebreak}Sarafa   Iyaniwura \newline
\mbox{}\quad  Mathematics \newline
\mbox{}\quad  University of British Columbia \newline
\mbox{}\quad  Canada \newline
\mbox{}\quad \href{mailto: iyaniwura@math.ubc.ca }{ iyaniwura@math.ubc.ca }
\end{absolutelynopagebreak}\vskip0.2cm
\begin{absolutelynopagebreak}Harsh   Jain \newline
\mbox{}\quad  Mathematics and Statistics \newline
\mbox{}\quad  University of Minnesota Duluth \newline
\mbox{}\quad  United States \newline
\mbox{}\quad \href{mailto: hjain@umn.edu }{ hjain@umn.edu }
\end{absolutelynopagebreak}\vskip0.2cm
\begin{absolutelynopagebreak}Ali   Karoobi \newline
\mbox{}\quad  Mathematics \newline
\mbox{}\quad  University of Manitoba \newline
\mbox{}\quad  Canada \newline
\mbox{}\quad \href{mailto: karoobia@myumanitoba.ca }{ karoobia@myumanitoba.ca }
\end{absolutelynopagebreak}\vskip0.2cm
\begin{absolutelynopagebreak}Marek   Kimmel \newline
\mbox{}\quad  Statistics \newline
\mbox{}\quad  Rice University \newline
\mbox{}\quad  United States \newline
\mbox{}\quad \href{mailto: kimmel@rice.edu }{ kimmel@rice.edu }
\end{absolutelynopagebreak}\vskip0.2cm
\begin{absolutelynopagebreak}Jude   Kong \newline
\mbox{}\quad  Mathematics and Statistics \newline
\mbox{}\quad  York University \newline
\mbox{}\quad  Canada \newline
\mbox{}\quad \href{mailto: jdkong@yorku.ca }{ jdkong@yorku.ca }
\end{absolutelynopagebreak}\vskip0.2cm
\begin{absolutelynopagebreak}Chapin   Korosec \newline
\mbox{}\quad  Mathematics and Statistics \newline
\mbox{}\quad  York University \newline
\mbox{}\quad  Canada \newline
\mbox{}\quad \href{mailto: chapwaite@gmail.com }{ chapwaite@gmail.com }
\end{absolutelynopagebreak}\vskip0.2cm
\begin{absolutelynopagebreak}Christopher   Kribs \newline
\mbox{}\quad  Mathematics \& Curriculum and Instruction \newline
\mbox{}\quad  University of Texas at Arlington \newline
\mbox{}\quad  United States \newline
\mbox{}\quad \href{mailto: kribs@uta.edu }{ kribs@uta.edu }
\end{absolutelynopagebreak}\vskip0.2cm
\begin{absolutelynopagebreak}Brandon   Legried \newline
\mbox{}\quad  Mathematics \newline
\mbox{}\quad  Georgia Institute of Technology \newline
\mbox{}\quad  United States \newline
\mbox{}\quad \href{mailto: blegried3@gatech.edu }{ blegried3@gatech.edu }
\end{absolutelynopagebreak}\vskip0.2cm
\begin{absolutelynopagebreak}Kang-Ling   Liao \newline
\mbox{}\quad  Mathematics \newline
\mbox{}\quad  University of Manitoba \newline
\mbox{}\quad  Canada \newline
\mbox{}\quad \href{mailto: Kang-Ling.Liao@umanitoba.ca }{ Kang-Ling.Liao@umanitoba.ca }
\end{absolutelynopagebreak}\vskip0.2cm
\begin{absolutelynopagebreak}Xiaochen   Long \newline
\mbox{}\quad  Department of Statistics \newline
\mbox{}\quad  Rice University \newline
\mbox{}\quad  United States \newline
\mbox{}\quad \href{mailto: xl81@rice.edu }{ xl81@rice.edu }
\end{absolutelynopagebreak}\vskip0.2cm
\begin{absolutelynopagebreak}Pedro   Lopez Gascon \newline
\mbox{}\quad  Mathematics \newline
\mbox{}\quad  University of Manitoba \newline
\mbox{}\quad  Canada \newline
\mbox{}\quad \href{mailto: lopezgap@myumanitoba.ca }{ lopezgap@myumanitoba.ca }
\end{absolutelynopagebreak}\vskip0.2cm
\begin{absolutelynopagebreak}Loïc   Louison \newline
\mbox{}\quad  Sciences et technologie \newline
\mbox{}\quad  Université de Guyane \newline
\mbox{}\quad  France \newline
\mbox{}\quad \href{mailto: loic.louison@univ-guyane.fr }{ loic.louison@univ-guyane.fr }
\end{absolutelynopagebreak}\vskip0.2cm
\begin{absolutelynopagebreak}Nadia   Loy \newline
\mbox{}\quad  DISMA-P.IVA 00518460019 \newline
\mbox{}\quad  Politecnico di Torino \newline
\mbox{}\quad  Italy \newline
\mbox{}\quad \href{mailto: nadia.loy@polito.it }{ nadia.loy@polito.it }
\end{absolutelynopagebreak}\vskip0.2cm
\begin{absolutelynopagebreak}Chinwendu Emilian   Madubueze \newline
\mbox{}\quad  Mathematics and Statistics \newline
\mbox{}\quad  York University \newline
\mbox{}\quad  Canada \newline
\mbox{}\quad \href{mailto: ce.madubueze@gmail.com }{ ce.madubueze@gmail.com }
\end{absolutelynopagebreak}\vskip0.2cm
\begin{absolutelynopagebreak}Anna   Marciniak-Czochra \newline
\mbox{}\quad  Institute of Applied Mathematics \newline
\mbox{}\quad  Heidelberg University \newline
\mbox{}\quad  Germany \newline
\mbox{}\quad \href{mailto: Anna.Marciniak@iwr.uni-heidelberg.de }{ Anna.Marciniak@iwr.uni-heidelberg.de }
\end{absolutelynopagebreak}\vskip0.2cm
\begin{absolutelynopagebreak}Solomon   Mensah \newline
\mbox{}\quad  Mathematics \newline
\mbox{}\quad  University of Manitoba  \newline
\mbox{}\quad  Canada \newline
\mbox{}\quad \href{mailto: mensahs2@myumanitoba.ca }{ mensahs2@myumanitoba.ca }
\end{absolutelynopagebreak}\vskip0.2cm
\begin{absolutelynopagebreak}Fabio   Milner \newline
\mbox{}\quad  Simon Levin MCMD Center \& School of Mathematical and Statistical Sciences \newline
\mbox{}\quad  Arizona State University \newline
\mbox{}\quad  United States \newline
\mbox{}\quad \href{mailto: fmilner@asu.edu }{ fmilner@asu.edu }
\end{absolutelynopagebreak}\vskip0.2cm
\begin{absolutelynopagebreak}Negar   Mohammadnejad \newline
\mbox{}\quad  Mathematics \newline
\mbox{}\quad  University of Manitoba \newline
\mbox{}\quad  Canada \newline
\mbox{}\quad \href{mailto: mohamm58@myumanitoba.ca }{ mohamm58@myumanitoba.ca }
\end{absolutelynopagebreak}\vskip0.2cm
\begin{absolutelynopagebreak}Jemal   Mohammed-Awel \newline
\mbox{}\quad  Mathematics \newline
\mbox{}\quad  Morgan State University \newline
\mbox{}\quad  United States \newline
\mbox{}\quad \href{mailto: jemal.mohammed-awel@morgan.edu }{ jemal.mohammed-awel@morgan.edu }
\end{absolutelynopagebreak}\vskip0.2cm
\begin{absolutelynopagebreak}Nicola   Mulberry \newline
\mbox{}\quad  Mathematics \newline
\mbox{}\quad  Simon Fraser \newline
\mbox{}\quad  Canada \newline
\mbox{}\quad \href{mailto: nicola\_mulberry@sfu.ca }{ nicola\_mulberry@sfu.ca }
\end{absolutelynopagebreak}\vskip0.2cm
\begin{absolutelynopagebreak}Toshiyuki   Namba \newline
\mbox{}\quad  Graduate School of Science \newline
\mbox{}\quad  Osaka Metropolitan University \newline
\mbox{}\quad  Japan \newline
\mbox{}\quad \href{mailto: tnamba@omu.ac.jp }{ tnamba@omu.ac.jp }
\end{absolutelynopagebreak}\vskip0.2cm
\begin{absolutelynopagebreak}Syeda Atika Batool   Naqvi \newline
\mbox{}\quad  Mathematics \newline
\mbox{}\quad  University of Manitoba \newline
\mbox{}\quad  Canada \newline
\mbox{}\quad \href{mailto: naqvisab@myumanitoba.ca }{ naqvisab@myumanitoba.ca }
\end{absolutelynopagebreak}\vskip0.2cm
\begin{absolutelynopagebreak}Jay   Newby \newline
\mbox{}\quad  Mathematical and Statistical Sciences \newline
\mbox{}\quad  University of Alberta \newline
\mbox{}\quad  Canada \newline
\mbox{}\quad \href{mailto: jnewby@ualberta.ca }{ jnewby@ualberta.ca }
\end{absolutelynopagebreak}\vskip0.2cm
\begin{absolutelynopagebreak}Hiroshi   Nishiura \newline
\mbox{}\quad  School of Public Health \newline
\mbox{}\quad  Kyoto University \newline
\mbox{}\quad  Japan \newline
\mbox{}\quad \href{mailto: nishiura.hiroshi.5r@kyoto-u.ac.jp }{ nishiura.hiroshi.5r@kyoto-u.ac.jp }
\end{absolutelynopagebreak}\vskip0.2cm
\begin{absolutelynopagebreak}Ryo   Oizumi \newline
\mbox{}\quad  Department of International Research and Cooperation \newline
\mbox{}\quad  National Institute of Population and Social Security Research \newline
\mbox{}\quad  Japan \newline
\mbox{}\quad \href{mailto: ooizumi-ryou@ipss.go.jp }{ ooizumi-ryou@ipss.go.jp }
\end{absolutelynopagebreak}\vskip0.2cm
\begin{absolutelynopagebreak}Lorenzo   Pellis \newline
\mbox{}\quad  Mathematics \newline
\mbox{}\quad  The University of Manchester \newline
\mbox{}\quad  United Kingdom \newline
\mbox{}\quad \href{mailto: lorenzo.pellis@manchester.ac.uk }{ lorenzo.pellis@manchester.ac.uk }
\end{absolutelynopagebreak}\vskip0.2cm
\begin{absolutelynopagebreak}Tin   Phan \newline
\mbox{}\quad  Theoretical Biology and Biophysics \newline
\mbox{}\quad  Los Alamos National Laboratory \newline
\mbox{}\quad  United States \newline
\mbox{}\quad \href{mailto: ttphan@lanl.gov }{ ttphan@lanl.gov }
\end{absolutelynopagebreak}\vskip0.2cm
\begin{absolutelynopagebreak}Tanya   Philippsen \newline
\mbox{}\quad  Mathematics and Statistics \newline
\mbox{}\quad  University of Victoria \newline
\mbox{}\quad  Canada \newline
\mbox{}\quad \href{mailto: tanya.philippsen@gmail.com }{ tanya.philippsen@gmail.com }
\end{absolutelynopagebreak}\vskip0.2cm
\begin{absolutelynopagebreak}Stephanie   Portet \newline
\mbox{}\quad  Mathematics \newline
\mbox{}\quad  University of Manitoba \newline
\mbox{}\quad  Canada \newline
\mbox{}\quad \href{mailto: stephanie.portet@umanitoba.ca }{ stephanie.portet@umanitoba.ca }
\end{absolutelynopagebreak}\vskip0.2cm
\begin{absolutelynopagebreak}Andrea   Pugliese \newline
\mbox{}\quad  Mathematics \newline
\mbox{}\quad  Università  degli Studi di Trento \newline
\mbox{}\quad  Italy \newline
\mbox{}\quad \href{mailto: andrea.pugliese@unitn.it }{ andrea.pugliese@unitn.it }
\end{absolutelynopagebreak}\vskip0.2cm
\begin{absolutelynopagebreak}Erica   Rutter \newline
\mbox{}\quad  Applied Mathematics \newline
\mbox{}\quad  University of California, Merced \newline
\mbox{}\quad  United States \newline
\mbox{}\quad \href{mailto: erutter2@ucmerced.edu }{ erutter2@ucmerced.edu }
\end{absolutelynopagebreak}\vskip0.2cm
\begin{absolutelynopagebreak}Paul   Salceanu \newline
\mbox{}\quad  Mathematics \newline
\mbox{}\quad  University of Louisiana at Lafayette \newline
\mbox{}\quad  United States \newline
\mbox{}\quad \href{mailto: salceanu@louisiana.edu }{ salceanu@louisiana.edu }
\end{absolutelynopagebreak}\vskip0.2cm
\begin{absolutelynopagebreak}Leili   Shahriyari \newline
\mbox{}\quad  Mathematics and Statistics \newline
\mbox{}\quad  University of Massachusetts Amherst \newline
\mbox{}\quad  United States \newline
\mbox{}\quad \href{mailto: lshahriyari@umass.edu }{ lshahriyari@umass.edu }
\end{absolutelynopagebreak}\vskip0.2cm
\begin{absolutelynopagebreak}Zhisheng   Shuai \newline
\mbox{}\quad  Mathematics \newline
\mbox{}\quad  University of Central Florida \newline
\mbox{}\quad  United States \newline
\mbox{}\quad \href{mailto: Zhisheng.Shuai@ucf.edu }{ Zhisheng.Shuai@ucf.edu }
\end{absolutelynopagebreak}\vskip0.2cm
\begin{absolutelynopagebreak}Nourridine   Siewe \newline
\mbox{}\quad  Mathematics \newline
\mbox{}\quad  Rochester Institute of Technology \newline
\mbox{}\quad  United States \newline
\mbox{}\quad \href{mailto: nxssma@rit.edu }{ nxssma@rit.edu }
\end{absolutelynopagebreak}\vskip0.2cm
\begin{absolutelynopagebreak}Stacey   Smith? \newline
\mbox{}\quad  Mathematics \newline
\mbox{}\quad  The University of Ottawa \newline
\mbox{}\quad  Canada \newline
\mbox{}\quad \href{mailto: stacey.smith@uottawa.ca }{ stacey.smith@uottawa.ca }
\end{absolutelynopagebreak}\vskip0.2cm
\begin{absolutelynopagebreak}Tracy   Stepien \newline
\mbox{}\quad  Mathematics \newline
\mbox{}\quad  University of Florida \newline
\mbox{}\quad  United States \newline
\mbox{}\quad \href{mailto: tstepien@ufl.edu }{ tstepien@ufl.edu }
\end{absolutelynopagebreak}\vskip0.2cm
\begin{absolutelynopagebreak}Yasuhiro   Takeushi \newline
\mbox{}\quad  Mathematical Sciences \newline
\mbox{}\quad  Aoyama Gakuin University \newline
\mbox{}\quad  Japan \newline
\mbox{}\quad \href{mailto: takeuchi@math.aoyama.ac.jp }{ takeuchi@math.aoyama.ac.jp }
\end{absolutelynopagebreak}\vskip0.2cm
\begin{absolutelynopagebreak}Ryan   Thiessen \newline
\mbox{}\quad  Mathematical and Statistical Sciences \newline
\mbox{}\quad  University of Alberta \newline
\mbox{}\quad  Canada \newline
\mbox{}\quad \href{mailto: rjt128@mail.usask.ca }{ rjt128@mail.usask.ca }
\end{absolutelynopagebreak}\vskip0.2cm
\begin{absolutelynopagebreak}Sonja   Tuerpitz \newline
\mbox{}\quad  Bioinformatics \newline
\mbox{}\quad  Friedrich Schiller University Jena \newline
\mbox{}\quad  Germany \newline
\mbox{}\quad \href{mailto: sonja.tuerpitz@uni-jena.de }{ sonja.tuerpitz@uni-jena.de }
\end{absolutelynopagebreak}\vskip0.2cm
\begin{absolutelynopagebreak}Necibe    Tuncer \newline
\mbox{}\quad  Mathematical Sciences \newline
\mbox{}\quad  Florida Atlantic University \newline
\mbox{}\quad  United States \newline
\mbox{}\quad \href{mailto: ntuncer@fau.edu }{ ntuncer@fau.edu }
\end{absolutelynopagebreak}\vskip0.2cm
\begin{absolutelynopagebreak}Pauline   van den Driessche \newline
\mbox{}\quad  Mathematics and Statistics \newline
\mbox{}\quad  University of Victoria \newline
\mbox{}\quad  Canada \newline
\mbox{}\quad \href{mailto: pvdd@math.uvic.ca }{ pvdd@math.uvic.ca }
\end{absolutelynopagebreak}\vskip0.2cm
\begin{absolutelynopagebreak}Marie Betsy   Varughese \newline
\mbox{}\quad  Mathematical and Statistical Sciences \newline
\mbox{}\quad  University of Alberta \newline
\mbox{}\quad  Canada \newline
\mbox{}\quad \href{mailto: mvarughe@ualberta.ca }{ mvarughe@ualberta.ca }
\end{absolutelynopagebreak}\vskip0.2cm
\begin{absolutelynopagebreak}Jorge   Velasco-Hernandez \newline
\mbox{}\quad  Instituto de Matemáticas \newline
\mbox{}\quad  UNAM \newline
\mbox{}\quad  Mexico \newline
\mbox{}\quad \href{mailto: jx.velasco@im.unam.mx }{ jx.velasco@im.unam.mx }
\end{absolutelynopagebreak}\vskip0.2cm
\begin{absolutelynopagebreak}Amy   Veprauskas \newline
\mbox{}\quad  Mathematics \newline
\mbox{}\quad  University of Louisiana at Lafayette \newline
\mbox{}\quad  United States \newline
\mbox{}\quad \href{mailto: amy.veprauskas@louisiana.edu }{ amy.veprauskas@louisiana.edu }
\end{absolutelynopagebreak}\vskip0.2cm
\begin{absolutelynopagebreak}Ren-Yi   Wang \newline
\mbox{}\quad  Statistics \newline
\mbox{}\quad  Rice University \newline
\mbox{}\quad  United States \newline
\mbox{}\quad \href{mailto: rw47@rice.edu }{ rw47@rice.edu }
\end{absolutelynopagebreak}\vskip0.2cm
\begin{absolutelynopagebreak}Xuyuan   Wang \newline
\mbox{}\quad  Mathematical and Statistical Science \newline
\mbox{}\quad  University of Alberta \newline
\mbox{}\quad  Canada \newline
\mbox{}\quad \href{mailto: xuyuan@ualberta.ca }{ xuyuan@ualberta.ca }
\end{absolutelynopagebreak}\vskip0.2cm
\begin{absolutelynopagebreak}Kenton   Watt \newline
\mbox{}\quad  Mathematics \newline
\mbox{}\quad  University of Manitoba \newline
\mbox{}\quad  Canada \newline
\mbox{}\quad \href{mailto: wattk2@myumanitoba.ca }{ wattk2@myumanitoba.ca }
\end{absolutelynopagebreak}\vskip0.2cm
\begin{absolutelynopagebreak}Adam   Wieler \newline
\mbox{}\quad  Mathematics \newline
\mbox{}\quad  University of Manitoba \newline
\mbox{}\quad  Canada \newline
\mbox{}\quad \href{mailto: wielera1@myumanitoba.ca }{ wielera1@myumanitoba.ca }
\end{absolutelynopagebreak}\vskip0.2cm
\begin{absolutelynopagebreak}Kathleen   Wilkie \newline
\mbox{}\quad  Mathematics \newline
\mbox{}\quad  Toronto Metropolitan University \newline
\mbox{}\quad  Canada \newline
\mbox{}\quad \href{mailto: kpwilkie@torontomu.ca }{ kpwilkie@torontomu.ca }
\end{absolutelynopagebreak}\vskip0.2cm
\begin{absolutelynopagebreak}Xiangye   Xu \newline
\mbox{}\quad  Mathematics \newline
\mbox{}\quad  University of Manitoba \newline
\mbox{}\quad  Canada \newline
\mbox{}\quad \href{mailto: xux8@myumanitoba.ca }{ xux8@myumanitoba.ca }
\end{absolutelynopagebreak}\vskip0.2cm
\begin{absolutelynopagebreak}Pei   Yuan \newline
\mbox{}\quad  Mathematics and Statistics \newline
\mbox{}\quad  York University \newline
\mbox{}\quad  Canada \newline
\mbox{}\quad \href{mailto: yuanp45@yorku.ca }{ yuanp45@yorku.ca }
\end{absolutelynopagebreak}\vskip0.2cm
\begin{absolutelynopagebreak}Veronika   Zarnitsyna \newline
\mbox{}\quad  Microbiology and Immunology \newline
\mbox{}\quad  Emory University \newline
\mbox{}\quad  United States \newline
\mbox{}\quad \href{mailto: veronika.i.zarnitsyna@emory.edu }{ veronika.i.zarnitsyna@emory.edu }
\end{absolutelynopagebreak}\vskip0.2cm
\begin{absolutelynopagebreak}Huaiping   Zhu \newline
\mbox{}\quad  Mathematics and Ststistics \newline
\mbox{}\quad  York University \newline
\mbox{}\quad  Canada \newline
\mbox{}\quad \href{mailto: huaiping@yorku.ca }{ huaiping@yorku.ca }
\end{absolutelynopagebreak}\vskip0.2cm



\end{document}
